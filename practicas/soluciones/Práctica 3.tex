\documentclass{article}
\usepackage{ifthen}
\usepackage{amssymb}
\usepackage{multicol}
\usepackage{graphicx}
\usepackage[absolute]{textpos}
\usepackage{amsmath, amscd, amssymb, amsthm, latexsym}
\usepackage{xspace,rotating,dsfont,ifthen}
\usepackage[spanish,activeacute]{babel}
\usepackage[utf8]{inputenc}
\usepackage{pgfpages}
\usepackage{pgf,pgfarrows,pgfnodes,pgfautomata,pgfheaps,xspace,dsfont}
\usepackage{listings}
\usepackage{multicol}
\usepackage{todonotes}
\usepackage{url}
\usepackage{float}
\usepackage{framed,mdframed}
\usepackage{cancel}

\usepackage[strict]{changepage}


\makeatletter


\newcommand\hfrac[2]{\genfrac{}{}{0pt}{}{#1}{#2}} %\hfrac{}{} es un \frac sin la linea del medio

\newcommand\Wider[2][3em]{% \Wider[3em]{} reduce los m\'argenes
\makebox[\linewidth][c]{%
  \begin{minipage}{\dimexpr\textwidth+#1\relax}
  \raggedright#2
  \end{minipage}%
  }%
}


\@ifclassloaded{beamer}{%
  \newcommand{\tocarEspacios}{%
    \addtolength{\leftskip}{4em}%
    \addtolength{\parindent}{-3em}%
  }%
}
{%
  \usepackage[top=1cm,bottom=2cm,left=1cm,right=1cm]{geometry}%
  \usepackage{color}%
  \newcommand{\tocarEspacios}{%
    \addtolength{\leftskip}{3em}%
    \setlength{\parindent}{0em}%
  }%
}

\usepackage{caratula}
\usepackage{enumerate}
\usepackage{hyperref}
\usepackage{graphicx}
\usepackage{amsfonts}
\usepackage{enumitem}
\usepackage{amsmath}

\decimalpoint
\hypersetup{colorlinks=true, linkcolor=black, urlcolor=blue}
\setlength{\parindent}{0em}
\setlength{\parskip}{0.5em}
\setcounter{tocdepth}{2} % profundidad de indice
\setcounter{section}{2} % nro de section
\renewcommand{\thesubsubsection}{\thesubsection.\Alph{subsubsection}}
\graphicspath{ {images/} }

% End latex config

\begin{document}

\titulo{Práctica 3}
\fecha{2do cuatrimestre 2021}
\materia{Álgebra I}
\integrante{Yago Pajariño}{546/21}{ypajarino@dc.uba.ar}

%Carátula
\maketitle
\newpage

%Indice
\tableofcontents
\newpage

% Aca empieza lo propio del documento
\section{Práctica 3}

\subsection{Ejercicio 1}

Por enunciado, $ A= \{ n \in V: n \geq 132 \} $

Y también, $ A^c = \{ n \in V: n < 132 \} $

Se que dado un elemento cualquiera, $ x \in V \iff (x \in \mathbb{N} \wedge x \bmod 15 = 0)$

Por lo tanto, $ A^c = \{ n \in V: (n < 132 \wedge n \bmod 15 = 0) \} $

Así, $ \#A^c = \lfloor \frac{132}{15} \rfloor = 8 $

Por extensión, $ A^c = \{ 15,30,45,60,75,90,105,120 \} $

\subsection{Ejercicio 2}

Defino el conjunto universal $ V = \{ n \in \mathbb{N}: n \leq 1000 \} $

Defino el conjunto $ T = \{ n \in \mathbb{N}: n \bmod 3 = 0 \} $

Defino el conjunto $ C = \{ n \in \mathbb{N}: n \bmod 5 = 0 \} $

Luego busco $ \#(T^c \cup C^c) = \#(T \cup C)^c $

Entonces $ (T \cup C) = \{ n \in \mathbb{N}: n \bmod 15 = 0 \} $ pues 3 y 5 son primos.

Por lo tanto $ \#(T \cup C) = \lfloor \frac{1000}{15} \rfloor = 66$

Y así, $ \#(T \cup C)^c = 1000 - 66 = 934 $

\subsection{Ejercicio 3}

$ \#(A \cup B \cup C) = \#A + \#B+ \#C - \#(A \cap B) - \#(A \cap C) - \#(B \cap C) + \#(A \cap B \cap C) $

\subsection{Ejercicio 4}
\subsubsection{Pregunta i}

Datos del enunciado:
\begin{enumerate}
    \item $ \#V = 150 $
    \item $ \#A = 83 $
    \item $ \#B = 67 $
    \item $ \#(A \cap B) = 45 $
\end{enumerate}

Luego,
\begin{align*}
    \#(A \cup B)^c &= \#V - \#(A\cup B) \\
    &= \#V - (\#A + \#B - \#(A \cap B)) \\
    &= 150 - (83 + 67 - 45) \\
    &= 45 \\
\end{align*}

\subsubsection{Pregunta ii}
TODO

\subsection{Ejercicio 5}

Datos del enunciado:
\begin{enumerate}
    \item Rutas BSAS - Ros = 3
    \item Rutas Ros - SF = 4
    \item Rutas SF - Req = 4
\end{enumerate}

Por lo tanto hay $ 3 \cdot 4 \cdot 2 = 24 $ formas de ir de Buenos Aires a Reconquista pasando por Rosario y Santa Fe.

\subsection{Ejercicio 6}
\subsubsection{Pregunta i}
Hay $ 8 \cdot 9\cdot 9\cdot 9 = 5832 $ números.

\subsubsection{Pregunta ii}
Calculando por el complemento:

Hay $ 9 \cdot 10\cdot 10\cdot 10 = 9000 $ números de cuatro cifras.

En el inciso anterior se calculó la cantidad de números que no tienen cierto dígito (calculado por 5, vale para 7).

Luego habrá $ 9000 - 5832 = 3168 $ números.

\subsection{Ejercicio 7}
Puede distribuirlos en $ 3^{17} $ formas.

\subsection{Ejercicio 8}

Defino $ A = \{ materias \}$, se que $ \#A = 5 $

Luego las posibles elecciones están dadas por $ \#P(A) = 2^5 = 32 $

Si tiene que cursar al menos dos materias, no puede elegir las opciones de cursar ninguna materia o una sola materia.

Así tiene $ 32 - 5 - 1 = 26 $ formas de cursar al menos dos materias.

\subsection{Ejercicio 9}

Se que A es de la forma $ A = \{ a_1, a_2, ... , a_n \} $

$R$ es una relación en $ A \times A \iff R \subseteq A \times A $: si $R$ es un subconjunto del producto cartesiano $ A \times A $

Luego la cantidad de relaciones en A será: $ \# P(A \times A) = 2^{n^2}$

\begin{enumerate}
    \item Reflexivas: $ 2^{n^2-2} $
    \item Simétricas: $ 2^{\sum_{k =1}^{n}k} = 2^{\frac{n(n+1)}{2}} $
    \item Simétricas: $ 2^{\sum_{k =1}^{n-1}k} = 2^{\frac{n(n-1)}{2}} $
\end{enumerate}

\subsection{Ejercicio 10}
\begin{enumerate}
    \item $ \#\{ f \in F / \text{f es función}\} = 12^5 $
    \item $ \#\{ f \in F / 10 \not \in \text{Im(f)} \} = 11^5 $
    \item $ \#\{ f \in F / 10 \in \text{Im(f)} \} = 12^5 - 11^5 $
    \item $ \#\{ f \in F / f(1) \in \{ 2,4,6 \} \} = 3 \cdot 12^4 $
\end{enumerate}

\subsection{Ejercicio 11}

\begin{enumerate}
    \item $7! = 5040$ funciones.
    \item $3! \cdot 4! = 144$ funciones.
\end{enumerate}

\subsection{Ejercicio 12}
De cinco cifras usando los dígitos $\{ 1,2,3,4,5 \}: 5!$

De cinco cifras usando los dígitos $\{ 1,2,3,4,5,6,7 \}: \frac{5!}{2!}$

De cinco cifras usando los dígitos $\{ 1,2,3,4,5,6,7 \}$ sin 2 en las cententas: $\frac{7!}{2!} \cdot \frac{4}{5}$

\subsection{Ejercicio 13}
Rdo. funciones inyectivas: Una función $f: A\rightarrow B$ es inyectiva sii $ (x \in A) \wedge (y \in A) \wedge (x\neq y) \implies f(x) \neq f(y)$ 

\begin{enumerate}
    \item $\frac{10!}{(10-7)!} = \frac{10!}{3!}$
    \item Para $f(1)$ tengo 5 opciones. Al resto todas menos las que ya fueron asignadas (9,8,7,...) $ \implies 5 \cdot \frac{9!}{3!}$
\end{enumerate}

\subsection{Ejercicio 14}
Defino $A = \{ 1,2,3,4,5,6,7 \}$ y $B = \{ 1,2,3,4,5,6,7 \}$

Luego $\#A = \#B = 7$

$f: A \rightarrow B \text{ es viyectiva } \iff \forall x \in A; \exists ! y \in B: f(x) = y$ \\
Y además me piden que $ f(\{ 1,2,3 \}) \subseteq \{ 3,4,5,6,7 \} $

Luego habrá $ \frac{5!}{2!} \cdot 4! $ funciones que cumplen lo pedido.

\subsection{Ejercicio 15}
Tengo $R$ relación de equivalencia en $A=\{ f: \{ 1,2,3,4 \} \rightarrow \{ 1,2,3,4,5,6,7,8 \}: \text{f es inyectiva} \}$

Por definición, $fRg \iff f(1) + f(2) = g(1) + g(2)$

Necesito saber cuantas $g \in A$ se relaciones con $ f(n) = n+2 $

Pero,
\begin{align*}
    fRg \iff f(1) + f(2) &= g(1) + g(2) \\
    3+4 &= g(1) + g(2) \\
    7 &= g(1) + g(2) \\
\end{align*}

Entonces, busco las $g \in A: g(1) + g(2) = 7$ 

Hay seis funciones de $\{ 1,2 \} \rightarrow \{ 2,3,4,5,6 \}$ que cumplen con esto.

Completo el total de funciones asignando el resto de los elementos de forma inyectiva.

Luego habrá $ 6 \cdot \frac{6!}{4!} = 180$ elementos dentro de la clase de equivalencia de $f(n) = n+2$

\subsection{Ejercicio 16}

Defino $A=\{ 1,2,3,...8 \}$ y $B=\{ 1,2,3,...,12 \}$ con $\#A = 8$ y $ \#B = 12 $

Condiciones que me piden:
\begin{enumerate}
    \item f inyectiva
    \item $f(5) + f(5) = 6$
    \item $ f(1) \leq 6 $
\end{enumerate}

Primero busco asignaciones a $f(5)$ y $f(6)$ que cumplan lo pedido. Para esto hay cuatro opciones posibles.

Luego $f(1)$ puede tomar cualquier valor menos los dos que ya fueron asignados ya que $f(5); f(6)$ siempre toman valores $\leq 6$. Luego para $f(1)$ hay 4 opciones.

Para los demás elementos de $A$ pueden tomar alguno de los 9 elementos restantes de $B$.

Por lo tanto hay $4 \cdot 4 \cdot \frac{9!}{4!} $ opciones.

\subsection{Ejercicio 17}
\begin{enumerate}
    \item $\binom{7}{4}$
    \item $\binom{6}{3}$
    \item $\binom{6}{4}$
    \item $\binom{5}{3} \cdot 2$
\end{enumerate}

\subsection{Ejercicio 18}
Por enunciado $A = \{ n \in \mathbb{N}: n \leq 20 \}$ y $\#A = 20$

\subsubsection{Pregunta i}
Defino $B_1 = \{ n \in \mathbb{N}: n \leq 20 \wedge n \bmod 3 = 0 \} = \{ 3,6,9,12,15,18 \}$

Luego para armar las funciones debo elegir 4 del conjunto $B_1$ y 6 elementos del conjunto $B - B_1$

Luego habrá $ \binom{6}{4} \cdot \binom{14}{6} $ subconjuntos.

\subsubsection{Pregunta ii}
Hay suma impar de dos elementos si uno de ellos es par y el otro impar. Entonces, todos los elementos deben ser pares o impares.

Si son todos pares $ \implies \binom{10}{5} $ subconjuntos.

Si son todos impares $ \implies \binom{10}{5} $ subconjuntos.

Luego habrá $ 2 \cdot \binom{10}{5} $

\subsection{Ejercicio 19}
Cada punto de una recta se une a dos de la otra para formar un triángulo. 

Es decir, para cada vértice en una recta, elijo dos en la otra recta para formar el triángulo.

Luego habrá $ \binom{m}{2} \cdot n $ con $ m \geq 2; n\in \mathbb{N} $

\subsection{Ejercicio 20}
Defino $ A = \{ 1,2,3,...,11 \} $ y $ B = \{ 1,2,3,...,16 \} $

Me piden:
\begin{enumerate}
    \item f inyectiva
    \item $ n, f(n) $ pares
    \item $ f(1) < f(3) < f(5) < f(7) $
\end{enumerate}

La segunda condición me dice que los pares solo pueden tener imagen par, luego habrá $ \#fp $ funciones para los pares.

$ \#fp = \frac{8!}{3!} $

Para los impares tengo que considerar la tercera condición, esta implica que no me importa el orden de los elementos de B, sino que me voy a quedar con aquel que cumple la condición.

Así habrá $\#fi$ funciones para los impares.

$\#fi = \binom{11}{4} \cdot 7 \cdot 6$

Por lo tanto, hay $ \frac{8!}{3!} \cdot \binom{11}{4} \cdot 7 \cdot 6 $ funciones que cumplen lo pedido.

\subsection{Ejercicio 21}
\begin{enumerate}
    \item $ 7! $
    \item $ \frac{7!}{3!} $
    \item $ \frac{12!}{3!\cdot 2!} $
\end{enumerate}

\subsection{Ejercicio 22}
\begin{enumerate}
    \item $ \binom{7}{3} \cdot 3! \cdot 4! $
    \item $ \binom{7}{4} \cdot 3! $
    \item $ 4! \cdot 4! $
\end{enumerate}

\subsection{Ejercicio 23}
\begin{enumerate}
    \item Por el complemento: $ 10! - \binom{10}{2} \cdot 8!$
    \item $ \binom{10}{3} \cdot 3! \cdot 7!$
\end{enumerate}

\subsection{Ejercicio 24}
Defino $ F = \{ D,D,D,D,D,D,N,N,B,P,H,K,C,M \} $

Condiciones:
\begin{enumerate}
    \item Dos frutas por día.
    \item No más de una N por día.
\end{enumerate}

Calculo por el complemento,

$ \# \text{Todas} - \# \text{Dos naranjas por día} = 14! - 7 \cdot 12!$

\end{document}

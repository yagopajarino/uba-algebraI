\documentclass{article}
\usepackage{ifthen}
\usepackage{amssymb}
\usepackage{multicol}
\usepackage{graphicx}
\usepackage[absolute]{textpos}
\usepackage{amsmath, amscd, amssymb, amsthm, latexsym}
\usepackage{xspace,rotating,dsfont,ifthen}
\usepackage[spanish,activeacute]{babel}
\usepackage[utf8]{inputenc}
\usepackage{pgfpages}
\usepackage{pgf,pgfarrows,pgfnodes,pgfautomata,pgfheaps,xspace,dsfont}
\usepackage{listings}
\usepackage{multicol}
\usepackage{todonotes}
\usepackage{url}
\usepackage{float}
\usepackage{framed,mdframed}
\usepackage{cancel}

\usepackage[strict]{changepage}


\makeatletter


\newcommand\hfrac[2]{\genfrac{}{}{0pt}{}{#1}{#2}} %\hfrac{}{} es un \frac sin la linea del medio

\newcommand\Wider[2][3em]{% \Wider[3em]{} reduce los m\'argenes
\makebox[\linewidth][c]{%
  \begin{minipage}{\dimexpr\textwidth+#1\relax}
  \raggedright#2
  \end{minipage}%
  }%
}


\@ifclassloaded{beamer}{%
  \newcommand{\tocarEspacios}{%
    \addtolength{\leftskip}{4em}%
    \addtolength{\parindent}{-3em}%
  }%
}
{%
  \usepackage[top=1cm,bottom=2cm,left=1cm,right=1cm]{geometry}%
  \usepackage{color}%
  \newcommand{\tocarEspacios}{%
    \addtolength{\leftskip}{3em}%
    \setlength{\parindent}{0em}%
  }%
}

\usepackage{caratula}
\usepackage{enumerate}
\usepackage{hyperref}
\usepackage{graphicx}
\usepackage{amsfonts}
\usepackage{enumitem}
\usepackage{amsmath}

\decimalpoint
\hypersetup{colorlinks=true, linkcolor=black, urlcolor=blue}
\setlength{\parindent}{0em}
\setlength{\parskip}{0.5em}
\setcounter{tocdepth}{2} % profundidad de indice
\setcounter{section}{4} % nro de section
\renewcommand{\thesubsubsection}{\thesubsection.\Alph{subsubsection}}
\graphicspath{ {images/} }

% End latex config

\begin{document}

\titulo{Práctica 5}
\fecha{2do cuatrimestre 2021}
\materia{Álgebra I}
\integrante{Yago Pajariño}{546/21}{ypajarino@dc.uba.ar}

%Carátula
\maketitle
\newpage

%Indice
\tableofcontents
\newpage

% Aca empieza lo propio del documento
\section{Práctica 5}

\subsection{Ejercicio 1}

\subsubsection{Pregunta i}

Busco los $ (a,b) \in \mathbb{Z}^2 $ tales que $ 7a+11b = 10 $

\textbf{Verifico que existe solución}

Dado que $ (7:11) = 1 \implies 1|10 \implies $ existe solución.

\textbf{Busco una solución particular}

Por propiedades del MCD se que existen $ (s,t) \in \mathbb{Z}^2 $ tales que:
\begin{align*}
    7.s + 11.t &= 1 \\
    7.(-3) + 11.2 &= 1 \\
    7.(-3).10 + 11.2.10 &= 1.10 \\
    7.(-30) + 11.20 &= 10 \\
    -210+220 &= 10 \\
\end{align*}

Luego $ (-30,20) $ es solución particular.

\textbf{Solución del homogeneo asociado}

$ 7.a+11.b = 0 \iff 7.a = -11.b \iff -11|7.a \iff -11|a \iff a = -11.k $

$ 7.a+11.b = 0 \iff 7.a = -11.b \iff 7(-11.k) = 11.b \iff b = 7.k$

Luego $ (-11.k, 7.k) $ es solución del homogeneo asociado, $ \forall k \in \mathbb{Z} $

\textbf{Solución general}

Uniendo las dos soluciones halladas previamente,

$ S = (-11.k, 7.k) + (-30,20) = (-11.k-30,7.k+20); \forall k \in \mathbb{Z} $

\textbf{Verifico}

Sea $ (a,b) = (-11.k-30,7.k+20) $ luego,
\begin{align*}
    7a+11b = 10 &\iff 7(-11.k-30) + 11(7.k+20) = 10\\ 
    &\iff 7(-11.k-30) + 11(7.k+20) = 10 \\
    &\iff 7.-11.k-210 + 11.7.k+220 = 10 \\
    &\iff -210 + 220 = 10 \\
\end{align*}

Verificado.

\subsubsection{Pregunta ii}

Busco los $ (a,b) \in \mathbb{Z}^2 $ tales que $ 20a+16b = 30 $

\textbf{Verifico que existe solución}

$ (20:16) = 4 \wedge 4\not | 30 $

Por lo tanto no hay solución en $ \mathbb{Z}^2 $ para la ecuación.

\subsubsection{Pregunta iii}

Busco los $ (a,b) \in \mathbb{Z}^2 $ tales que $ 39a-24b = 6 $

\textbf{Verifico que existe solución}

$ (39:24) = 3 \wedge 3|6 $ luego existe solución en $ \mathbb{z}^2 $

\textbf{Coprimizar}

Dado que el MCD es disinto a 1, debo coprimizar la ecuación para no perder soluciones.

$ 20a+16b = 30 \leftrightsquigarrow 13a-8b = 2$

\textbf{Busco una solución particular}

$ 13.(2) - 8.(3) = 26-24 = 2 $

Luego $ (2,3) $ es solución particular.

\textbf{Solución del homogeneo asociado}

$ 13a -8b = 0 \iff 13a = 8b \implies 13|8b \iff 13|b \iff b = 13k $

$ 13a -8b = 0 \iff 13a = 8b \iff 13a = 8(13k) \iff a = 8k $

Luego $ (8k,13k) $ es solución del homogeneo asociado, $ \forall k \in \mathbb{Z} $

\textbf{Solución general}

Uniendo las dos soluciones halladas previamente,

$ S = (8k,13k) + (2,3) = (8k+2, 13k+3); \forall k \in \mathbb{Z} $

\textbf{Verifico}

Sea $ (a,b) = (8k+2, 13k+3) $ luego,
\begin{align*}
    39a-24b = 6 &\iff 39(8k+2)-24(13k+3) = 6\\ 
    &\iff 39.8k+78-24.13k-72 = 6\\ 
    &\iff 78-72 = 6\\ 
\end{align*}

Verificado.

\subsubsection{Pregunta iv}

Busco los $ (a,b) \in \mathbb{Z}^2 $ tales que $ 1555a-300b = 11 $

\textbf{Verifico que existe solución}

$ (1555:300) = 5 \wedge 4\not | 5 $

Por lo tanto no hay solución en $ \mathbb{Z}^2 $ para la ecuación.

\subsection{Ejercicio 2}

Primero busco soluciones $ (a,b) \in \mathbb{Z}^2 $ para la ecuación $ 33a+9b=120 $

\textbf{Verifico que existe solución}

$ (33:9) = 3 \wedge 3|120 \implies $ existe solución.

\textbf{Coprimizar}

Dado que el MCD es disinto a 1, debo coprimizar la ecuación para no perder soluciones.

$ 33a+9b = 120 \leftrightsquigarrow 11a+3b=40$

\textbf{Busco una solución particular}

Por propiedades del MCD se que existen $ (s,t) \in \mathbb{Z}^2 $ tales que:
\begin{align*}
    11.s + 3.t &= 1 \\
    11.(2) + 3.(-7) &= 1 \\
    11.2.(40) + 3.(-7).(40) &= 1.(40) \\
    11.80 + 3.(-280) &= 40 \\
\end{align*}
Luego $ (80,-280) $ es solución particular.

\textbf{Solución del homogeneo asociado}

$ 11a +3b = 0 \iff 11a = -3b \implies 11|-3b \implies 11|b \implies b=11k $

$ 11a +3b = 0 \iff 11a = -3b \implies 11a = -3(11k) \implies a = -3k $

Luego $ (-3k,11k) $ es solución del homogeneo asociado, $ \forall k \in \mathbb{Z} $

\textbf{Solución general}

Uniendo las dos soluciones halladas previamente,

$ S = (-3k,11k) + (80,-280) = (-3k+80, 11k-280); \forall k \in \mathbb{Z} $

Luego tengo definidas las restricciones para a y b de tal forma que cumplan con la diofántica, ahora uso los datos de divisibilidad.

$ a = -3k+80 \implies -3k+80 \equiv 0(4) \implies k \equiv 0(4)$

$ b = 11k-280 \implies 11k-280 \equiv 0(8) \implies k \equiv 0(8)$

Pero $ k \equiv 0(8) \iff k \equiv 0(4) $

Luego $ k = 8n \implies a = 3(8n) + 80 \wedge b=11(8n)-280; n \in \mathbb{Z}$

Rta.: $ (a,b) = (24n+80, 88n-280); \forall n \in \mathbb{Z} $

\subsection{Ejercicio 3}

Busco los $ (a,b) \in \mathbb{Z}^2 $ que cumplen $ 39a+48b=135 $

\textbf{Verifico que existe solución}

$ (39:135) = 3 \wedge 3|135 \implies $ existe solución.

\textbf{Coprimizar}

Dado que el MCD es disinto a 1, debo coprimizar la ecuación para no perder soluciones.

$ 39a+48b=135 \leftrightsquigarrow 13a+16b=45$

\textbf{Busco una solución particular}

$ (225,-180) $ es solución particular.

\textbf{Solución del homogeneo asociado}

$ 13a +16b = 0 \iff 13a = -16b \implies 13|-16b \implies 13|b \implies b=13k $

$ 13a +16b = 0 \iff 13a = -16b \implies 13a = -16(13k) \implies a = -16k $

Luego $ (-16k,13k) $ es solución del homogeneo asociado, $ \forall k \in \mathbb{Z} $

\textbf{Solución general}

Uniendo las dos soluciones halladas previamente,

$ S = (-16k,13k) + (225,-180) = (-16k+225,13k-180); \forall k \in \mathbb{Z} $

Luego, \\
$ a \geq 0 \implies -16k+225 \geq 0 \implies 16k \leq 225 \implies k \leq \frac{225}{16} \implies k \leq 14 $ \\
$ b \geq 0 \implies 13k-180 \geq 0 \implies 13k \geq 180 \implies k \geq \frac{180}{13} \implies k \geq 14 $

Luego $ 14 \leq k \leq 14 \implies k = 14 \implies $ se compran 1 unidad de a y 2 de b, gastando 135 pesos.

\subsection{Ejercicio 4}

\begin{enumerate}
    \item $ 17x \equiv 3(11) \iff 6x \equiv 3(11) \iff 2.6 \equiv 2.3(11) \iff x \equiv 6(11) $
    \item $ 56x \equiv 28(35) \iff 21x \equiv 28(35) \iff 3x \equiv 4(5) \iff 6x \equiv 8(5) \iff x \equiv 3(5) $
    \item $ 56x \equiv 2(884) $ No tiene solución pues $ (56:884) = 4 \not | 2 $
    \item $ 78x \equiv 30(12126) \iff 13x\equiv 5(2021) \iff 311.13x \equiv 311.5(2021) \iff x \equiv 1551(2021) $
\end{enumerate}

\subsection{Ejercicio 5}

Primero resuelvo la diofántica: busco $ (a,b) \in \mathbb{Z}^2: 28a+10b = 26 $

\textbf{Verifico que existe solución}

$ (28:10) = 2 \wedge 2|26 \implies $ existe solución.

\textbf{Coprimizar}

Dado que el MCD es disinto a 1, debo coprimizar la ecuación para no perder soluciones.

$ 28a+10b =26 \leftrightsquigarrow 14a+5b=13$

\textbf{Busco una solución particular}

$ (-13,39) $ es solución particular.

\textbf{Solución del homogeneo asociado}

$ 14a+5b = 0 \iff 14a = -5b \implies 14|-5b \implies 14|b \implies b=14k $

$ 14a+5b = 0 \iff 14a = -5b \implies 14a = -5(14k) \implies a = -5k $

Luego $ (-5k, 14k) $ es solución del homogeneo asociado, $ \forall k \in \mathbb{Z} $

\textbf{Solución general}

Uniendo las dos soluciones halladas previamente,

$ S = (-5k, 14k) + (-13,39) = (-5k-13, 14k+39); \forall k \in \mathbb{Z} $

Luego $ a = -5-13 \wedge b= 14k+39 $. Usando el dato de la congruencia,
\begin{align*}
    b\equiv 2a(5) &\iff 14k+39 \equiv 2(-5k-13)(5) \\
    &\iff 4k+4 \equiv 4(5) \\
    &\iff 4(k+1) \equiv 4(5) \\
    &\iff k+1 \equiv 1(5) \\
    &\iff k \equiv 0(5) \\
\end{align*}
Luego se que $ k = 5n; n \in \mathbb{Z} $

Por lo tanto, las soluciones serán $ (a,b) = (-25n-13, 70n+39 ); n \in \mathbb{Z} $

\subsection{Ejercicio 6}

\begin{align*}
    7a \equiv 5(18) &\iff (-5).7.a \equiv (-5).5(18) \\  
    &\iff -35a \equiv -25(18) \\
    &\iff a \equiv 11(18)
\end{align*}
Luego el resto de dividir a $a$ por 18 es 11.

\subsection{Ejercicio 7}
Primero busco los $ (x,y) \in \mathbb{Z}^2: 110x+250y=100 $

\textbf{Verifico que existe solución}

$ (110:250) = 10 \wedge 10|100 \implies $ existe solución.

\textbf{Coprimizar}

Dado que el MCD es disinto a 1, debo coprimizar la ecuación para no perder soluciones.

$ 110x+250y =100 \leftrightsquigarrow 11x+25y=10$

\textbf{Busco una solución particular}

$ (-90,40) $ es solución particular.

\textbf{Solución del homogeneo asociado}

$ 11x+25y = 0 \iff 11x = -25y \implies 11|-25y \implies 11|y \implies y=11k $

$ 11x+25y = 0 \iff 11x = -25y \implies 11x=-25(11k) \implies x = -25k $

Luego $ (-25k, 11k) $ es solución del homogeneo asociado, $ \forall k \in \mathbb{Z} $

\textbf{Solución general}

Uniendo las dos soluciones halladas previamente,

$ S = (-25k, 11k) + (-90,40) = (-25k-90, 11k+40); \forall k \in \mathbb{Z} $

Luego $ x = -25k-90 \wedge y= 11k+40 $,
\begin{align*}
    37^2 |(x-y)^{4321} &\iff 37^2 | (-25k-90-11k-40)^{4321} \\
    &\iff 37^2 | (-36k-130)^{4321} \\
\end{align*}

Pero por propiedades de la divisibilidad,
\begin{align*}
    37^2 | (-36k-130)^{4321} &\iff 37|(-36k-130) \\
    &\iff -36k \equiv 130(37) \\
    &\iff k \equiv 4(37) \\
\end{align*}

Por lo tanto $ (x, y) = (-25n-90, 11n+40) $ para todo $ n \equiv 4(37) $

\end{document}

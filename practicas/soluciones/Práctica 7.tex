\documentclass{article}
\usepackage{ifthen}
\usepackage{amssymb}
\usepackage{multicol}
\usepackage{graphicx}
\usepackage[absolute]{textpos}
\usepackage{amsmath, amscd, amssymb, amsthm, latexsym}
\usepackage{xspace,rotating,dsfont,ifthen}
\usepackage[spanish,activeacute]{babel}
\usepackage[utf8]{inputenc}
\usepackage{pgfpages}
\usepackage{pgf,pgfarrows,pgfnodes,pgfautomata,pgfheaps,xspace,dsfont}
\usepackage{listings}
\usepackage{multicol}
\usepackage{todonotes}
\usepackage{url}
\usepackage{float}
\usepackage{framed,mdframed}
\usepackage{cancel}

\usepackage[strict]{changepage}


\makeatletter


\newcommand\hfrac[2]{\genfrac{}{}{0pt}{}{#1}{#2}} %\hfrac{}{} es un \frac sin la linea del medio

\newcommand\Wider[2][3em]{% \Wider[3em]{} reduce los m\'argenes
\makebox[\linewidth][c]{%
  \begin{minipage}{\dimexpr\textwidth+#1\relax}
  \raggedright#2
  \end{minipage}%
  }%
}


\@ifclassloaded{beamer}{%
  \newcommand{\tocarEspacios}{%
    \addtolength{\leftskip}{4em}%
    \addtolength{\parindent}{-3em}%
  }%
}
{%
  \usepackage[top=1cm,bottom=2cm,left=1cm,right=1cm]{geometry}%
  \usepackage{color}%
  \newcommand{\tocarEspacios}{%
    \addtolength{\leftskip}{3em}%
    \setlength{\parindent}{0em}%
  }%
}

\usepackage{caratula}
\usepackage{enumerate}
\usepackage{hyperref}
\usepackage{graphicx}
\usepackage{amsfonts}
\usepackage{enumitem}
\usepackage{amsmath}

\decimalpoint
\hypersetup{colorlinks=true, linkcolor=black, urlcolor=blue}
\setlength{\parindent}{0em}
\setlength{\parskip}{0.5em}
\setcounter{tocdepth}{2} % profundidad de indice
\setcounter{section}{6} % nro de section
\renewcommand{\thesubsubsection}{\thesubsection.\Alph{subsubsection}}
\graphicspath{ {images/} }

% End latex config

\begin{document}

\titulo{Práctica 7}
\fecha{2do cuatrimestre 2021}
\materia{Álgebra I}
\integrante{Yago Pajariño}{546/21}{ypajarino@dc.uba.ar}

%Carátula
\maketitle
\newpage

%Indice
\tableofcontents
\newpage

% Aca empieza lo propio del documento
\section{Práctica 7}

\subsection{Ejercicio 1}

Rdo. propiedades del producto y suma de polinomios: 
\begin{itemize}
    \item Grado de un producto de polinomios $ gr(ab) = gr(a) + gr(b) $
    \item Coeficiente principal de un prodcuto de polinomios $ cp(ab) = ca(a)\cdot cd(b) $
    \item $ \begin{cases}
        gr(f+g) \leq max(gr(f); gr(g)) \\
        gr(f+g) = max(gr(f); gr(g)) \iff gr(f) \neq gr(g) \vee (gr(f) = gr(g) \wedge cp(f) \neq cp(g))\\
    \end{cases}  $
\end{itemize}

\subsubsection{Pregunta i}

\begin{itemize}
    \item $ gr(p) = 77 . gr(4x^6-2x^5+3x^2-2x+7) = 77.6 = 462 $
    \item $ cp(p) = 4^{77} $ 
\end{itemize}

\subsubsection{Pregunta ii}

Sea $ p = a^4 - b^7 $ con $ a = -3x^7 + 5x^3 + x^2 - x + 5 $ y $ b = 6x^4 + 2x^3 + x - 2 $
\begin{align*}
    gp(p) &= max(gr(a^4); gr(b^7)) \iff gr(a^4) \neq gr(b^7) \vee cp(a^4) \neq cp(b^7) \\
    &= max(7.4; 4.7) \iff cp(a^4) \neq cp(b^7) \\
    &= 28 \iff (-3)^4 \neq 6^7 \\
    &= 28 \iff 81 \neq 279936
\end{align*}
\begin{itemize}
    \item $ gr(p) = 28 $
    \item $ cp(p) = 81-6^7 $ 
\end{itemize}

\subsubsection{Pregunta iii}

Sea $ p = a - b + c $ con $ \begin{cases}
    a = (-3x^5 + x^4 - x + 5)^4 \\
    b = 82x^{20} \\
    c = 19x^{19}
\end{cases} $

Luego $ p = 81x^{20} + (...) - 81x^{20}+19x^{19} \implies gr(p) = 19 $ pues se cancelan los termino con $ x^{20} $

Entonces busco el coeficiente para $ x^{19} $
\begin{align*}
    cp(p) &= a_{19} + b_{19} + c_{19} \\
    &= (-3.-3.-3.1) + 0 + 19 \\
    &= -27 + 0 + 19 \\
    &= -8 \\
\end{align*}
\begin{itemize}
    \item $ gr(p) = 19 $
    \item $ cp(p) = -8 $ 
\end{itemize}

\subsection{Ejercicio 2}

\begin{enumerate}
    \item \begin{enumerate}
        \item En $ \mathbb{Q}[x] = 2 $
        \item En $ \frac{\mathbb{Z}}{2\mathbb{Z}}[x] = 2 $
    \end{enumerate}
    \item Usando bin de Newton, $ c(20) = \binom{133}{20}(3i)^{113} $
    \item Usando bin de Newton cuatro veces, \begin{itemize}
        \item $ a_1 = \binom{4}{1}x^{1}(-1)^3 \cdot \binom{19}{19}x^{19}5^0 = -4x^{20} $
        \item $ a_2 = \binom{4}{2}x^{2}(-1)^2 \cdot \binom{19}{18}x^{18}5^1 = 570x^{20} $
        \item $ a_3 = \binom{4}{3}x^{3}(-1)^1 \cdot \binom{19}{17}x^{17}5^2 = -17100x^{20} $
        \item $ a_4 = \binom{4}{4}x^{4}(-1)^0 \cdot \binom{19}{16}x^{16}5^3 = 121125x^{20} $
    \end{itemize}
    Luego $ c(20) = a_1 + a_2 + a_3 + a_4 - 5 = -4+570-17100+121125 -5 = 104586 $
    \item $ c(20) = 21504 $
\end{enumerate}

\subsection{Ejercicio 3}

\subsubsection{Pregunta i}

Reescribo el polinomio que me dan,
\begin{align*}
    f^2 = xf + x + 1 &\iff f^2 - xf = x+1 \\
    &\iff f(f-x) = x+1 \\
    &\iff f\neq 0 \wedge f-x \neq 0 \\
\end{align*}
Tomo grado a ambos lados,
\begin{align*}
    gr(f) + gr(f-x) &= gr(x+1) \\
    gr(f) + gr(f-x) &= 1 \\
\end{align*}
Luego el grado de f tiene que se menor a 2.

\textbf{Caso gr(f) = 1}

Si $ gr(f) = 1 \implies gr(f-x) = 0 $ para cumplir la igualdad de grados.

Luego f es de la forma $ f = ax+b $ con $ a = 1 $

Entonces,
\begin{align*}
    f(f-x) = x+1 &\iff (x+b)(x+b-x) = x+1 \\
    &\iff xb + b^2 = x+1 \\
    &\iff \text{Por igualdad de polinomios} \begin{cases}
        b = 1 \\
        b^2 = 1 \\
    \end{cases} \iff b = 1
\end{align*}
Así, $ f_1 = x+1 $

\textbf{Caso gr(f) = 0}

Que el grado del polinomio sea igual a cero implica que $ f = c $ con c una constante.

Entonces,
\begin{align*}
    f(f-x) = x+1 &\iff c(c-x) = x+1 \\
    &\iff c^2 - cx = x+1 \\
    &\iff \text{Por igualdad de polinomios} \begin{cases}
        -c = 1 \\
        c^2 = 1
    \end{cases} \implies c = -1
\end{align*}
Así, $ f_2 = -1 $

Rta.: $ f = x+1 $ y $ f=-1 $

\subsubsection{Pregunta ii}

Reescribo el polinomio que me dan,
\begin{align*}
    f^2 - xf = x^2 + 1 \iff f(f-x) = -x^2 + 1
\end{align*}
Tomo grado a ambos lados de la igualdad.
\begin{align*}
    gr(f) + gr(f-x) &= gr(-x^2 + 1) \\
    0 + 2 &= 2 \text{ No puede ser} \\
    1 + 1 &= 2 \\
    2 + 0 &= 2 \text{ No puede ser} \\
\end{align*}
Así, el único caso posible es que $ gr(f) = 1 $ y que $ gr(f-x) = 1 $

Sea $ f = ax+b $,
\begin{align*}
    f(f-x) = -x^2 + 1 &\iff (ax+b)(ax+b-x) = -x^2+1 \\
    &\iff (ax+b)((a-1)x+b) = -x^2+1 \\
    &\iff a(a-1)x^2 + abx + b(a-1)x + b^2 = -x^2+1 \\
    &\iff a(a-1)x^2 + (ab+b(a-1))x + b^2 = -x^2+1 \\
    &\iff \text{Por igualdad de polinomios }\begin{cases}
        a(a-1) = -1 \\
        ab+b(a-1) = 0 \\
        b^2 = -1 
    \end{cases}
\end{align*}
Busco soluciones para el sistema de tres ecuaciones que resultó.

De la tercera, se que $ b = \pm 1 $

$ b = 1 \implies a + a + 1 = 0 \iff 2a = 1 \iff a = \frac{1}{2} $

Pero con $ a = \frac{1}{2} \wedge b= 1 \implies \frac{1}{2} (\frac{1}{2}-1) = \frac{1}{4} - \frac{1}{2} = -\frac{1}{4} \neq -1 $

Luego $ b = 1 $ NO sirve.

$ b = -1 \implies -a-a+1 = 0 \implies -2a = -1 \implies a = \frac{1}{2} $

Se llega al mismo valor de a que con b=1 y ya se probó que no sirve.

Por lo tanto, $ \not \exists f \in \mathbb{C}[x] $ que cumpla lo pedido.

\subsubsection{Pregunta iii}

Reescribo el polinomio que me dan,
\begin{align*}
    (x+1)f^2 = x^6 + xf &\iff (x+1)f^2 - xf = x^6 \\
    &\iff f((x+1)f - x) = x^6 \\
\end{align*}
Aplico grado a ambos lados de la igualdad.
\begin{align*}
    gr(f) + gr((x+1)f - x) &= gr(x^6) \\
    0 + 6 &= 6 \\
    1 + 5 &= 6 \\
    2 + 4 &= 6 \\
    3 + 3 &= 6 \\
    4 + 2 &= 6 \\
    5 + 1 &= 6 \\
    6 + 0 &= 6 \\
\end{align*}
Luego de dar todos los posibles valores a $ gr(f) $, se puede ver que no existe $ gr((x+1)f - x) $ que cumpla lo pedido.

Por lo tanto, $ \not \exists f \in \mathbb{C}[x] $ que cumpla lo pedido.

\subsubsection{Pregunta iv}

Dado que por enunciado se que $ f \neq 0 $, puedo reescribir la igualdad como,
\begin{align*}
    f^3 = gr(f) \cdot x^2f &\iff f^2 = gr(f)\cdot x^2 \\
\end{align*}
Aplico grado a ambos lados de la igualdad.
\begin{align*}
    gr(f^2) &= gr(gr(f)\cdot x^2) \\
    gr(f^2) &= 2 \\
    gr(f.f) &= 2 \\
    2gr(f) &= 2 \\
    gr(f) &= 1 \\
\end{align*}
Luego, con $ f = ax+b $,
\begin{align*}
    f^2 = gr(f)\cdot x^2 &\iff (ax+b)^2 = x^2 \\
    &\iff a^2x^2 + 2abx + b^2 = x^2 \\
    &\iff \text{Por igualdad de polinomios} \begin{cases}
        a^2 = 1 \\
        2ab = 0 \\
        b = 0
    \end{cases}
\end{align*}
Entonces, $ a = \pm 1 $ y $ b = 0 $ son las soluciones del sistema.

Rta.: $ f_1 = x $ y  $ f_2 = -x $ son los únicos polinomios que cumplen lo pedido.

\subsection{Ejercicio 4}

En estos ejercicios hay que hacer la división con la caja. Yo voy a dejar los resultados de cada paso de la división.

\subsubsection{Pregunta i}

\begin{enumerate}
    \item C = $ 5x^2 $; R = $ 2x^3-10x^2-x+4 $ 
    \item C = $ 2x $; R = $ -10x^2-5x+4 $ 
    \item C = $ -10 $; R = $ -5x-16 $ 
\end{enumerate}

Rta: \begin{itemize}
    \item C = $ 5x^2 + 2x - 10 $
    \item R = $ -5x-16 $
\end{itemize}

\subsubsection{Pregunta ii}

\begin{enumerate}
    \item C = $ 2x^2 $; R = $ x^3-2x^2-4 $ 
    \item C = $ \frac{1}{2}x $; R = $ -2x^2 - \frac{1}{2}x - 4 $ 
    \item C = $ -1 $; R = $ -\frac{1}{2}x - 3 $ 
\end{enumerate}

Rta: \begin{itemize}
    \item C = $ 2x^2 + \frac{1}{2}x - 1 $
    \item R = $ -\frac{1}{2}x - 3 $
\end{itemize}

\subsubsection{Pregunta iii}

\begin{enumerate}
    \item C = $ x^{n-1} $; R = $ x^{n-1} - 1 $ 
    \item C = $ x^{n-2} $; R = $ x^{n-2} - 1 $ 
    \item C = $ ... $; R = $ ... $ 
    \item C = $ 1 $; R = $ 0 $ 
\end{enumerate}

Rta: \begin{itemize}
    \item C = $ \sum_{i = 1}^{n}x^{n-i}(x-1) $
    \item R = $ 0 $
\end{itemize}

\subsection{Ejercicio 5}

\subsubsection{Pregunta i}

Haciendo $ x^3 + 2x^2 + 2x + 1 $ dividido $ x^2 + ax + 1 $ llego a:
\begin{align*}
    x^3 + 2x^2 + 2x + 1 = (x^2 + ax + 1)(x + 2 - a) + \left( 1 - 2a + a^2 \right)x -1 + a
\end{align*}
Luego busco que el resto $ \left( 1 - 2a + a^2 \right)x -1 + a $ sea igual a cero, por igualdad de polinomios,
\begin{align*}
    \left( 1 - 2a + a^2 \right)x -1 + a = 0 \iff \begin{cases}
    1 - 2a + a^2 = 0 \\
    -1+a = 0
    \end{cases}
\end{align*}
De la segunda ecuación se que $ -1 + a = 0 \iff a = 1 $.

Reemplazando en la primera y verifico que $ a = 1 $ cumple lo pedido, $ 1-2a+a^2 = 1-2+1 = 0 $

Rta.: $ a = 1 $

\subsubsection{Pregunta ii}

Haciendo $ x^4 - ax^3 + 2x^2 + x + 1 $ dividido $ x^2 + x + 1 $ llego a:
\begin{align*}
    x^4 - ax^3 + 2x^2 + x + 1 = (x^2 + x + 1)(x^2 + (-a-1)x + 2 + a) + 1-2-a
\end{align*}
Luego busco que el resto $ 1-2-a $ sea igual a cero, por igualdad de polinomios,
\begin{align*}
    1-2-a = 0 \iff \begin{cases}
        1-2-a = 0 \\
    \end{cases}
\end{align*}
Por lo tanto $ a = -1 $ es el único que lo cumple. 

\subsubsection{Pregunta iii}

(Esta división es larga y pesada)
Haciendo $ x^5 - 3x^3 - x^2 - 2x + 1 $ dividido $ x^2 + ax + 1 $ llego a:
\begin{align*}
    x^5 - 3x^3 - x^2 - 2x + 1 = (x^2 + ax + 1)(x^3 - ax^2 + (-4a^2)x + (-1+5a-a^3)) + \left[ (2-a^2 + a - 5a^2 + a^4)x + (2-5a + a^3) \right]
\end{align*}
Luego busco que el resto sea igual a $ -8x + 4 $,
\begin{align*}
    \left[ (2-a^2 + a - 5a^2 + a^4)x + (2-5a + a^3) \right] = -8x + 4 \iff \begin{cases}
        a^4 - 6a^2 + a + 2 = -8 \\
        a^3 - 5a + 2 = 4
    \end{cases}
\end{align*}
De la segunda ecuación,
\begin{align*}
    a^3 - 5a + 2 = 4 &\iff a^3 - 5a - 2 = 0 \\
    &\iff a(a^2 - 5) - 2 = 0 \\
\end{align*}
A simple vista veo que $ a = -2 $ es solución, luego usando Ruffini,
\begin{align*}
    a^3 - 5a - 2 = (a^2 - 2a - 1)(a+2)
\end{align*}
Por lo tanto busco soluciones para $ a^2 - 2a - 1 = 0 \iff a = \frac{2 \pm \sqrt[]{8}}{2} $

Queda ver cuales de estos valores de a hallados cumple la primer ecuación.

\begin{itemize}
    \item $ a = -2 \implies a^4 - 6a^2 + a + 2 = 16 - 24 - 2 + 2 = -8 $
    \item $ a = \frac{2 + \sqrt[]{8}}{2} \implies a^4 - 6a^2 + a + 2 \neq -8 $ (Wolfram) 
    \item $ a = \frac{2 - \sqrt[]{8}}{2} \implies a^4 - 6a^2 + a + 2 \neq -8 $ (Wolfram) 
\end{itemize}

\subsection{Ejercicio 6}
TODO

\subsection{Ejercicio 7}

\subsubsection{Pregunta i}

Por definición de congruencias se que $ x^{31} - 2 \equiv 0(x^{31} - 2) \implies x^{31} \equiv 2(x^{31} - 2) $

Luego,
\begin{align*}
    x^{353} - x - 1 &\equiv (x^{31})^11 \cdot x^{12} - x - 1 (x^{31} - 2) \\
    &\equiv 2^11 \cdot x^{12} - x - 1 (x^{31} - 2) \\
    &\equiv 2048x^{12} - x - 1 (x^{31} - 2)\\
\end{align*}
Rta.: $ r_{x^{31} - 2}(x^{353} - x - 1) = 2048x^{12} - x - 1 $

\subsubsection{Pregunta ii}

Se que $ x^{6} + 1 | x^6 + 1 \iff x^6 + 1 \equiv 0 (x^6 + 1) \iff x^6 \equiv -1 (x^6 + 1) $

Luego,
\begin{align*}
    x^{1000} + x^{40} + x^{20} + 1 &\equiv (x^6)^{166} \cdot x^4 + (x^6)^6 \cdot x^4 + (x^6)^3 \cdot x^2 + 1 \\
    &\equiv (-1)^{166} \cdot x^4 + (-1)^6 \cdot x^4 + (-1)^3 \cdot x^2 + 1 \\
    &\equiv x^4 + x^4 - x^2 + 1 \\
    &\equiv 2x^4 - x^2 + 1 \\
\end{align*}
Rta.: $ r_{x^6 + 1}(x^{1000} + x^{40} + x^{20} + 1) = 2x^4 - x^2 + 1 $

\subsubsection{Pregunta ii}

Se que $ x^{100} - x + 1 \equiv 0 (x^{100} - x + 1) \iff x^{100} \equiv x-1(x^{100} -x + 1) $

Luego,
\begin{align*}
    x^{200} - 3x^{101} + 2 &\equiv (x^{100})^2 - 3(x^{100})x + 2 \\
    &\equiv (x-1)^2 - 3(x-1)x + 2 \\
    &\equiv x^2 - 2x + 1 - 3x^2 + 3x + 2 \\
    &\equiv -2x^2 + x + 3 (x^{100} -x + 1) \\
\end{align*}
Rta.: $ r_{x^{100} -x + 1}(x^{200} - 3x^{101} + 2) = -2x^2 + x + 3 $

\subsection{Ejercicio 8}
TODO

\subsection{Ejercicio 9}

Se resuelve con el algoritmo de Euclides en polinomios. Calculadora de MCD de polinomios \url{https://planetcalc.com/7760/}

\begin{enumerate}
    \item \begin{itemize}
        \item $ MCD = -x+1 $
    \end{itemize}
    \item \begin{itemize}
        \item $ MCD = x^2 + 1 $
        \item $ x^2 + 1 = f + (-x^3)g $
    \end{itemize}
    \item \begin{itemize}
        \item $ MCD = 3 $
        \item $ 3 = (-x+2)f + (1+2x^2 - 4x)g $
    \end{itemize}
\end{enumerate}

\subsection{Ejercicio 10}

Se que el resto tiene que tener grado menor al divisor, luego $ gr(r) \leq 2 $

Por algoritmo de división de polinomios existen $ q $ cociente y $ r $ resto tales que:
\begin{align*}
    f = q(x^3-2x^2-x+2) + r \\
\end{align*}
El enunciado me da las evaluaciones de $f$ en $ 1; 2; -1 $, luego
\begin{align*}
    f(1) &= q(1)(1-2-1+2) + r(1) \implies f(1) = r(1) = -2 \\ 
    f(2) &= q(1)(8-8-2+2) + r(2) \implies f(2) = r(2) = 1 \\ 
    f(-1) &= q(1)(-1-2+1+2) + r(-1) \implies f(-1) = r(-1) = 0 \\ 
\end{align*}
Por lo tanto se que $ r $ es de la forma $ ax^2 + bx + c $, con $ a;b;c \in \mathbb{Q} $, luego

$ \begin{cases}
    r(1) = a + b + c = -2 \\
    r(2) = 4a + 2b + c = 1 \\
    r(-1) = a - b + c = 0 \\
\end{cases} $

Restando la tercera a la primera, $ 2b = -2 \iff b = -1 $

Rearmando el sistema con lo hallado,

$ \begin{cases}
    a + c = -1 \\
    4a + c = 3 \\
    a + c = -1
\end{cases} $

La tercera es igual a la primera así que la puedo eliminar y restando la primera a la segunda:

$ 3a = 4 \iff a = \frac{4}{3} $

Luego $ a + c = -1 \iff \frac{4}{3} + c = -1 \iff c = -\frac{7}{3} $

Así, $ r_{x^3-2x^2-x+2}(f) = \frac{4}{3}x^2 - x - \frac{7}{3} $

\subsection{Ejercicio 11}

Sea $ f = x^{2n} + 3x^{n+1} + 3x^n - 5x^2 + 2x + 1 $ \\
Sea $ g = x^3 - x $

Se que $ f = q.g + r $ con $ gr(r) \leq 2 \implies r = ax^2 + bx + c $

Busco raíces de g,
\begin{align*}
    x^3 - x = 0 &\iff x(x^2 - 1) = 0  \\
    &\iff x \in \{ -1,0,1 \}
\end{align*}
Evalúo f para las raíces halladas,
\begin{align*}
    f(0) &= q(0)g(0) + r(0) \implies r(0) = 1 \\
    f(1) &= q(1)g(1) + r(1) \implies r(1) = 1+3+3-5+2+1 = 5 \\ 
    f(-1) &= q(-1)g(-1) + r(-1) \implies r(-1) = 1+3(-1)^{n+1} + 3(-1)^n -5 - 2 - 1 = -5 \\ 
\end{align*}
Por lo tanto, sabiendo que $ r(x) = ax^2 + bx + c $
$ \begin{cases}
    r(0) = 1 = c \\
    r(1) = 5 = a+b+c \\
    r(-1) = -5 = 25a - 5b + c
\end{cases} $

Sabiendo $ c = 1 \implies \begin{cases}
    a+b = 4 \implies a = 4-b \\
    25a-5b = -6
\end{cases} $

Sabiendo $ a = 4-b \implies $
\begin{align*}
    25(4-b) - 5b &= -6 \\
    100 - 25b - 5b &= -6 \\
    -30b &= -106 \\
    b &= \frac{53}{15}
\end{align*}

Luego $ a = 4-b \implies a = 4- \frac{53}{15} = \frac{7}{15}$

Así, $ r_{f}(g) = \frac{7}{15}x^2 + \frac{53}{15}x +1 $

\subsection{Ejercicio 12}

Sea $ w = x^3 $ \\
Sea $ g = w^2 + w - 2 $

Busco raíces de g
\begin{align*}
    g(w) = 0 \iff w = \frac{-1 \pm \sqrt[]{1-4.1.-2}}{2} \iff \begin{cases}
        w_1 = 1 \\
        w_2 = -2
    \end{cases}
\end{align*}
Por lo tanto, recordando que $ w = x^3 $,
\begin{align*}
    w_1 = x^3 = 1 \iff \begin{cases}
        x_1 = 1 \\
        x_2 = -\frac{1}{2} + \frac{\sqrt[]{3}}{2}i \\
        x_3 = -\frac{1}{2} - \frac{\sqrt[]{3}}{2}i \\
    \end{cases}
\end{align*}
Y,
\begin{align*}
    w_2 = x^3 = -2 \iff \begin{cases}
        x_4 = -\sqrt[3]{2} \\
        x_5 = \frac{\sqrt[3]{2}}{2} + \sqrt[3]{2} \cdot \frac{\sqrt[]{3}}{2}i \\
        x_6 = \frac{\sqrt[3]{2}}{2} - \sqrt[3]{2} \cdot \frac{\sqrt[]{3}}{2}i \\
    \end{cases}
\end{align*}

\subsection{Ejercicio 13}

Por definición de raíz, $ w + w^2 + w^4 $ es raíz de f $ \iff f(w + w^2 + w^4) = 0 $

Luego se que,
\begin{itemize}
    \item $ w = e^{\frac{2}{7} \pi i} $
    \item $ w^2 = e^{\frac{4}{7} \pi i} $
    \item $ w^4 = e^{\frac{8}{7} \pi i} $
\end{itemize}
Luego defino,
\begin{itemize}
    \item $ r = Re(w + w^2 + w^4) = \cos{\frac{2}{7} \pi} + \cos{\frac{4}{7} \pi} + \cos{\frac{8}{7} \pi} = -\frac{1}{2} $
    \item $ m = Im(w + w^2 + w^4) = \sin{\frac{2}{7} \pi} + \sin{\frac{4}{7} \pi} + \sin{\frac{8}{7} \pi} = \frac{\sqrt[]{7}}{2} $
\end{itemize}
Luego evalúo f en $ k = r + m.i $
\begin{align*}
    f(k) &= (r + m.i)^2 + (r + m.i) + 2 \\
    &= (-\frac{1}{2} + \frac{\sqrt[]{7}}{2}.i)^2 + (-\frac{1}{2} + \frac{\sqrt[]{7}}{2}.i) + 2 \\
    &= \frac{1}{4} - \frac{\sqrt[]{7}}{2}.i - \left(\frac{\sqrt[]{7}}{2}\right)^2 - \frac{1}{2} + \frac{\sqrt[]{7}}{2}.i + 2 \\
    &= 0 \\
\end{align*}

\subsection{Ejercicio 14}

\subsubsection{Pregunta i}

Me piden probar que $ (w+w^{-1}) $ y $ (w^2 + w^{-2}) $ son raíces de f.
\begin{align*}
    w+w^{-1} &= e^{\frac{2}{5} \pi i} + e^{-\frac{2}{5} \pi i} \\
    &= \cos{\frac{2}{5} \pi} + \cos{-\frac{2}{5} \pi} + i\left( \sin{\frac{2}{5} \pi} + \sin{-\frac{2}{5} \pi} \right) \\
    &= \cos{(\frac{2}{5} \pi)} + \cos{(-\frac{2}{5} \pi)} \\
\end{align*}
Luego sea $ A = \cos{(\frac{2}{5} \pi)} + \cos{(-\frac{2}{5} \pi)} \implies f(A) = A^2 + A - 1 = 0 $ 

Así, $ (w+w^{-1}) $ es raíz de f.
\begin{align*}
    w^2 + w^{-2} &= e^{\frac{4}{5} \pi i} + e^{-\frac{4}{5} \pi i} \\
    &= \cos{\frac{4}{5} \pi} + \cos{-\frac{4}{5} \pi} + i\left( \sin{\frac{4}{5} \pi} + \sin{-\frac{4}{5} \pi} \right) \\
    &= \cos{(\frac{4}{5} \pi)} + \cos{(-\frac{4}{5} \pi)} \\
\end{align*}
Luego sea $ B = \cos{(\frac{4}{5} \pi)} + \cos{(-\frac{4}{5} \pi)} \implies f(B) = B^2 + B - 1 = 0 $ 

Así, $ (w^2 + w^{-2}) $ es raíz de f.

\subsubsection{Pregunta ii}
TODO

\subsection{Ejercicio 15}
\subsubsection{Pregunta i}

$a$ es raíz de $f \iff (x-a)|f \iff f = n(x-a) $ \\
$a$ es raíz de $g \iff (x-a)|g \iff f = m(x-a) $ 

Por propiedades del MCD se que existen s, t tales que,
\begin{align*}
    (f:g) = sf + tg &\iff (f:g) = sn(x-a) + tm(x-a) \\
    &\iff (f:g) = (x-a) \cdot (sn + tm) \\
    &\iff (x-a)|(f:g) \\
\end{align*}
Luego $ (x-a)|(f:g) \iff (x-a) $ es raíz de $ (f:g) $ como se quería probar.

\subsubsection{Pregunta ii}

Primero busco el MCD entre $ x^4 + 3x - 2 $ y $ x^4 + 3x^3 - 3x + 1 $

(Acá van las cuentas del algo de Euclides)

Luego $ MCD = x^2 + x - 1 $

Busco raíces del MCD,
\begin{align*}
    x^2 + x - 1 = 0 &\iff x = \frac{-1 \pm \sqrt[]{1-4.1.-1}}{2} \\
    &\iff x = \frac{-1 \pm \sqrt[]{5}}{2} \\
\end{align*}
Luego que f teng auna raíz común con g $ \implies (f:g)|f $
\begin{align*}
    x^2 + x - 1 | x^4 + 3x^3 - 3x + 1 \iff x^4 + 3x^3 - 3x + 1 = q(x^2 + x - 1)
\end{align*}
(Acá va la división)

Obtengo que, $ x^4 + 3x^3 - 3x + 1 = (x^2 - x + 2)(x^2 + x - 1) $

Ahora busco raíces de $ x^2 - x + 2 $
\begin{align*}
    x^2 - x + 2 = 0 &\iff x = \frac{1 \pm \sqrt[]{-7}}{2} \\
    &\iff x = \frac{1 \pm \sqrt[]{7}i}{2} \\
\end{align*}
Luego las raíces de f son:
\begin{itemize}
    \item $ x_1 = \frac{-1 + \sqrt[]{5}}{2} $
    \item $ x_2 = \frac{-1 - \sqrt[]{5}}{2} $
    \item $ x_3 = \frac{1 + \sqrt[]{7}i}{2} $
    \item $ x_4 = \frac{1 - \sqrt[]{7}i}{2} $
\end{itemize}

\subsection{Ejercicio 16}

\subsubsection{Pregunta i}
La idea es evaluar en la función y sus derivadas hasta encontrar la derivada en la que no vale cero.
\begin{align*}
    f(1) &= 1-2+1 = 0 \implies mult(1,f) \geq 1 \\
    f'(x) &= 5x^4 - 6x^2 + 1 \\
    f'(1) &= 5-6+1 = 0 \implies mult(1,f) \geq 2 \\
    f''(x) &= 20x^3 - 12x \\
    f''(1) &= 20-12 \neq 0 \implies mult(1,f) = 2
\end{align*}

\subsubsection{Pregunta ii}
\begin{align*}
    f(x) &= x^6 - 3x^4 + 4 \\
    f(i) &= (i^2)^3 - 3(i^2)^2 + 4 = -1-3+4 = 0 \implies mult(i,f) \geq 1 \\
    f'(x) &= 6x^5 - 12x^3 \\
    f'(i) &= 6(i^2)^2 . i - 12(i^2).i = 6i+12i \neq 0 \implies mult(i,f) = 1 \\
\end{align*}

\subsubsection{Pregunta iii}
\begin{align*}
    f(x) &= (x-2)^2 \cdot (x^2-4) + (x-2)^3 \cdot (x-1) \\
    f(2) &= 0 + 0 = 0 \implies mult(2,f) \geq 1 \\
    f'(x) &= 2(x-2) \cdot (x^2-4) + (x-2)^2 \cdot 2x + 3(x-2)^2 \cdot (x-1) + (x-2)^3 \\
    f'(x) &= 8x^3 -33x^2 +36x -4 \\
    f'(2) &= 0 + 0 + 0 + 0 = 0 \implies mult(2,f) \geq 2 \\
    f''(x) &= 24x^2 -66x +36 \\
    f''(2) &= 96 -264 +36 = 0 \implies mult(2,f) \geq 3 \\
    f'''(x) &= 48x - 66 \\
    f'''(2) &= 96 - 66 = 30 \neq 0 \implies mult(2,f) = 3 \\
\end{align*}

\subsection{Ejercicio 17}
Por propiedades de la multiplicidad de raíces, se que $ f $ tiene raíces simples si $ \forall a \in \mathbb{C}: f(a) = 0 \implies f'(a) \neq 0 $

Luego,
\begin{align*}
    f'(x) &= (n+1)n x^n -n(n+1)x^{n-1} \\
    &= (n+1)n x^{n-1} (x-1) = 0 \iff x \in \{ 0,1 \} \\
\end{align*}
Por lo tanto las unicas posibles raíces multiples son $ x_0 = 0; x_1 = 1 $

Queda ver los valores de a tales que $f(x_1)$, $f(x_2)$ son iguales a cero,
\begin{align*}
    f(0) = a = 0 \iff a = 0 \\
    f(1) = n-(n+1) + a = 0 \iff a = n+1-n = 1
\end{align*}
Rta.: $f$ tiene raíces simples $ \iff a \in \{ 0, 1 \} $

\subsection{Ejercicio 18}
Por propiedades de la multiplicidad de raíces, se que $ f $ tiene raíces multiples si $ \exists a \in \mathbb{C}: f(a) = 0 \wedge f'(a) = 0 $

Luego,
\begin{align*}
    f'(x) &= (2n+1)x^{2n} - (2n+1) \\
    &= (2n+1)(x^{2n} - 1) = 0 \iff x = \pm 1 \\
\end{align*}
Entonces busco los a tales que $f(1) = 0$ y $f(-1) = 0$

\begin{align*}
    f(1) &= 1-(2n+1) +a = 0 \iff a = 2n \iff a \equiv 0(2)\\
    f(-1) &= (-1)^{2n+1}-(2n+1)(-1) +a = -1+2n+1+a = 0 \iff a = -2n \iff a \equiv 0(2) \\
\end{align*}
Rta.: Tiene raíces multiples $ \forall a \in \mathbb{C}: a \equiv 0(2) $

\subsection{Ejercicio 19}
Al igual que en los anteriores, primero busco la derivada y busco los valores para los que es igual a cero.
\begin{align*}
    f'(x) = 2x^{19} + 80x^9 &= 0 \\
    2x^{9}(x^{10} + 40) &= 0 \iff (x = 0) \vee (x^{10} = -40) \\
\end{align*}
Luego $ x = 0 $ es raíz multiple de f sii $a = 0$ y tiene multiplicidad 10.

Ahora veo el caso $ x^{10} = -40 $
\begin{align*}
    f(x) = (x^{10})^2 +8x^{10} + 2a \\
    f(x) = (-40)^2 +8(-40) + 2a \iff a = 640 \\
\end{align*}
Luego si $ a = 640 $, los x tales que $ x^{10} = -40 $ serán raíces multiples de f, dado que $gr(f) = 20 \implies $ f tiene 20 raíecs en $\mathbb{C}$. Dado que existen 10 x tales que $x^{10} = -40 \implies $ f tiene 10 raíces de multiplicidad 2.  

\subsection{Ejercicio 20}
Por propiedades de las raíces multiples, se que $f$ tiene raíz múltiple $ \iff \exists \alpha \in \mathbb{C}: f(\alpha) = 0 \wedge f'(\alpha) = 0 $

Luego busco los x tales que $f'(x) = 0$,
\begin{align*}
    &f'(x) = 68x^{67} - 68x^{3} = 68x^3(x^{64}-1) \\
    \implies &f'(x) = 0 \iff (x=0) \vee (x^{64}=1)
\end{align*}
Si $x=0$, $ f(0) = 16 \neq 0 $. Luego no es raíz de f.

Si $x^{64} = 1$,
\begin{align*}
    f(x) = x^{64} \cdot x^4 - 17x^4 - 16 &= 0 \\
    x^4 - 17x^4 - 16 &= 0 \\
    x^4(1-17) &= 16 \\
    x^4 &= -1 \\
\end{align*}
Luego los $\alpha$ que cumplen lo pedido son $ \{\alpha \in \mathbb{C}: \alpha \in G_{64} \wedge \alpha^4=-1 \} $

Pero se que,
\begin{align*}
    a^{64} = (a^{4})^{16} = (-1)^{16} = 1
\end{align*}
Luego si $ a^{4} = -1 \implies a \in G_{64} $

Y los $ \alpha $ tales que $ \alpha^4 = -1 $ son:
\begin{itemize}
    \item $ \alpha_0 = e^{\frac{1}{4}\pi i} $
    \item $ \alpha_1 = e^{\frac{3}{4}\pi i} $
    \item $ \alpha_2 = e^{\frac{5}{4}\pi i} $
    \item $ \alpha_3 = e^{\frac{7}{4}\pi i} $
\end{itemize}
Y cada uno de ellos tiene $ mult(\alpha_i, f) = 2$

\subsection{Ejercicio 21}

Para este ejercicio se puede usar la regla de la división de Ruffini, 
\href{https://calculadorasonline.com/calculadora-de-division-sintetica-regla-de-ruffini-online/}{Calculadora de Ruffini}
\subsubsection{Pregunta i}
Probado usando Ruffini

\subsubsection{Pregunta ii}
Usando ruffini queda resto igual a $ a+2 $, luego f es divisible por $(x-1)^3 \iff a+2 = 0 \iff a = -2$

\subsection{Ejercicio 22}

Busco $a$ tales que
\begin{itemize}
    \item $ f(1) = 0 $
    \item $ f'(1) = 0 $
    \item $ f''(1) \neq 0 $
\end{itemize}
Luego,
\begin{align*}
    f(1) &= 1-a-3+2+3a-2a = 0; \forall a \in \mathbb{C} \implies mult(1, f) \geq 1 \\
    f'(x) &= 4x^3 - 3ax^2  -6x + 2 + 3a \\
    f'(1) &= 4-3a+6+2+3a = 0; \forall a \in \mathbb{C} \implies mult(1, f) \geq 2 \\
    f''(x) &= 12x^2 - 6ax + 6 \\
    f''(1) &= 12-6a-6 = 12-6a
\end{align*}
Luego $f(1) \neq 0 \iff 12-6a \neq 0 \iff 12 = 6a \iff a = 2 $

Rta.: 1 es raíz doble de f para $ a = 2 $

\subsection{Ejercicio 23}

f tiene raíces simples $ \iff \forall a \in \mathbb{C}: f(a) = 0 \implies f'(a) \neq 0 $

Si defino $ f_n = \sum_{k = 0}^{n}\frac{x^k}{k!} $

Luego $ f' = \sum_{k = 1}^{n}\frac{kx^{k-1}}{k!} = \sum_{k = 0}^{n-1}\frac{x^{k-1}}{(k-1)!} = f_{n-1} $

Ahora supongo $ f_n(a) = 0 $ y quiero ver que $ f_{n-1}(a) \neq 0 $

Pero, $ f_{n-1}(a) = f_n(a) - \frac{\alpha^n}{n!} = 0 - \frac{\alpha^n}{n!} \neq 0; \forall \alpha \neq 0 $

Y si $ \alpha = 0 \implies f_n(0) = 1 \neq 0 \implies x = 0 $ no es raíz de $ f_n $ como se quería probar.

\subsection{Ejercicio 24}

Demostración usando inducción

Defino $ p(n): f_n(i) = 0 \wedge f'_n(i) = 0 \wedge f''_n(i) \neq 0 $

\textbf{Caso base n = 1}
\begin{align*}
    f_1(i) = 1-2+1 = 0 \\
    f'_1(x) = 4x^3+4x \\
    f'_1(i) = -4i + 4i = 0 \\
    f''_1(x) = 12x^2 + 4 \\
    f''_1(i) = -12+4 \neq 0 \\
\end{align*}
Luego $ p(1) $ es verdadero.

\textbf{Paso inductivo}

Dado $ h \geq 1 $ quiero probar que $ p(h) \implies p(h+1) $

HI: $ f_h(i) = 0 \wedge f'_h(i) = 0 \wedge f''_h(i) \neq 0 $

Qpq: $ f_{h+1}(i) = 0 \wedge f'_{h+1}(i) = 0 \wedge f''_{h+1}(i) \neq 0 $

Pero,
\begin{align*}
    &f_{h+1} = (x-i)(f_h + f'_h) \implies f_{h+1}(i) = (i-i)(f_h(i) + f'_h(i)) = 0 \\
    &f'_{h+1} = (f_h + f'_h) + (x-i)(f'_h + f''_h) \implies f'_{h+1}(i) = 0 + (i-i)(f'_h + f''_h) = 0 \\
    &f''_{h+1} = f'_h + f''_h + f'_h + f''_h + (x-i)(f'_h + f''_h) \implies f''_{h+1}(i) = 0 + f''_h + 0 + f''_h + 0 = 2f''_h \neq 0
\end{align*}
Luego $ p(h) \implies p(h+1); \forall h \geq 1 $

Por lo tanto $ p(n) $ es verdadero, $ \forall n \in \mathbb{N} $

\subsection{Ejercicio 25}
TODO

\subsection{Ejercicio 26}
TODO

\end{document}

\documentclass{article}
\usepackage{ifthen}
\usepackage{amssymb}
\usepackage{multicol}
\usepackage{graphicx}
\usepackage[absolute]{textpos}
\usepackage{amsmath, amscd, amssymb, amsthm, latexsym}
\usepackage{xspace,rotating,dsfont,ifthen}
\usepackage[spanish,activeacute]{babel}
\usepackage[utf8]{inputenc}
\usepackage{pgfpages}
\usepackage{pgf,pgfarrows,pgfnodes,pgfautomata,pgfheaps,xspace,dsfont}
\usepackage{listings}
\usepackage{multicol}
\usepackage{todonotes}
\usepackage{url}
\usepackage{float}
\usepackage{framed,mdframed}
\usepackage{cancel}

\usepackage[strict]{changepage}


\makeatletter


\newcommand\hfrac[2]{\genfrac{}{}{0pt}{}{#1}{#2}} %\hfrac{}{} es un \frac sin la linea del medio

\newcommand\Wider[2][3em]{% \Wider[3em]{} reduce los m\'argenes
\makebox[\linewidth][c]{%
  \begin{minipage}{\dimexpr\textwidth+#1\relax}
  \raggedright#2
  \end{minipage}%
  }%
}


\@ifclassloaded{beamer}{%
  \newcommand{\tocarEspacios}{%
    \addtolength{\leftskip}{4em}%
    \addtolength{\parindent}{-3em}%
  }%
}
{%
  \usepackage[top=1cm,bottom=2cm,left=1cm,right=1cm]{geometry}%
  \usepackage{color}%
  \newcommand{\tocarEspacios}{%
    \addtolength{\leftskip}{3em}%
    \setlength{\parindent}{0em}%
  }%
}

\usepackage{caratula}
\usepackage{enumerate}
\usepackage{hyperref}
\usepackage{graphicx}
\usepackage{amsfonts}
\usepackage{enumitem}
\usepackage{amsmath}

\decimalpoint
\hypersetup{colorlinks=true, linkcolor=black, urlcolor=blue}
\setlength{\parindent}{0em}
\setlength{\parskip}{0.5em}
\setcounter{tocdepth}{2} % profundidad de indice
\setcounter{section}{6} % nro de section
\renewcommand{\thesubsubsection}{\thesubsection.\Alph{subsubsection}}
\graphicspath{ {images/} }

% End latex config

\begin{document}

\titulo{Práctica 7}
\fecha{2do cuatrimestre 2021}
\materia{Álgebra I}
\integrante{Yago Pajariño}{546/21}{ypajarino@dc.uba.ar}

%Carátula
\maketitle
\newpage

%Indice
\tableofcontents
\newpage

% Aca empieza lo propio del documento
\section{Práctica 7}

\subsection{Ejercicio 1}

Rdo. propiedades del producto y suma de polinomios: 
\begin{itemize}
    \item Grado de un producto de polinomios $ gr(ab) = gr(a) + gr(b) $
    \item Coeficiente principal de un prodcuto de polinomios $ cp(ab) = ca(a)\cdot cd(b) $
    \item $ \begin{cases}
        gr(f+g) \leq max(gr(f); gr(g)) \\
        gr(f+g) = max(gr(f); gr(g)) \iff gr(f) \neq gr(g) \vee (gr(f) = gr(g) \wedge cp(f) \neq cp(g))\\
    \end{cases}  $
\end{itemize}

\subsubsection{Pregunta i}

\begin{itemize}
    \item $ gr(p) = 77 . gr(4x^6-2x^5+3x^2-2x+7) = 77.6 = 462 $
    \item $ cp(p) = 4^{77} $ 
\end{itemize}

\subsubsection{Pregunta ii}

Sea $ p = a^4 - b^7 $ con $ a = -3x^7 + 5x^3 + x^2 - x + 5 $ y $ b = 6x^4 + 2x^3 + x - 2 $
\begin{align*}
    gp(p) &= max(gr(a^4); gr(b^7)) \iff gr(a^4) \neq gr(b^7) \vee cp(a^4) \neq cp(b^7) \\
    &= max(7.4; 4.7) \iff cp(a^4) \neq cp(b^7) \\
    &= 28 \iff (-3)^4 \neq 6^7 \\
    &= 28 \iff 81 \neq 279936
\end{align*}
\begin{itemize}
    \item $ gr(p) = 28 $
    \item $ cp(p) = 81-6^7 $ 
\end{itemize}

\subsubsection{Pregunta iii}

Sea $ p = a - b + c $ con $ \begin{cases}
    a = (-3x^5 + x^4 - x + 5)^4 \\
    b = 82x^{20} \\
    c = 19x^{19}
\end{cases} $

Luego $ p = 81x^{20} + (...) - 81x^{20}+19x^{19} \implies gr(p) = 19 $ pues se cancelan los termino con $ x^{20} $

Entonces busco el coeficiente para $ x^{19} $
\begin{align*}
    cp(p) &= a_{19} + b_{19} + c_{19} \\
    &= (-3.-3.-3.1) + 0 + 19 \\
    &= -27 + 0 + 19 \\
    &= -8 \\
\end{align*}
\begin{itemize}
    \item $ gr(p) = 19 $
    \item $ cp(p) = -8 $ 
\end{itemize}

\end{document}

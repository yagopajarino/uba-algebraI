\documentclass{article}
\usepackage{ifthen}
\usepackage{amssymb}
\usepackage{multicol}
\usepackage{graphicx}
\usepackage[absolute]{textpos}
\usepackage{amsmath, amscd, amssymb, amsthm, latexsym}
\usepackage{xspace,rotating,dsfont,ifthen}
\usepackage[spanish,activeacute]{babel}
\usepackage[utf8]{inputenc}
\usepackage{pgfpages}
\usepackage{pgf,pgfarrows,pgfnodes,pgfautomata,pgfheaps,xspace,dsfont}
\usepackage{listings}
\usepackage{multicol}
\usepackage{todonotes}
\usepackage{url}
\usepackage{float}
\usepackage{framed,mdframed}
\usepackage{cancel}

\usepackage[strict]{changepage}


\makeatletter


\newcommand\hfrac[2]{\genfrac{}{}{0pt}{}{#1}{#2}} %\hfrac{}{} es un \frac sin la linea del medio

\newcommand\Wider[2][3em]{% \Wider[3em]{} reduce los m\'argenes
\makebox[\linewidth][c]{%
  \begin{minipage}{\dimexpr\textwidth+#1\relax}
  \raggedright#2
  \end{minipage}%
  }%
}


\@ifclassloaded{beamer}{%
  \newcommand{\tocarEspacios}{%
    \addtolength{\leftskip}{4em}%
    \addtolength{\parindent}{-3em}%
  }%
}
{%
  \usepackage[top=1cm,bottom=2cm,left=1cm,right=1cm]{geometry}%
  \usepackage{color}%
  \newcommand{\tocarEspacios}{%
    \addtolength{\leftskip}{3em}%
    \setlength{\parindent}{0em}%
  }%
}

\usepackage{caratula}
\usepackage{enumerate}
\usepackage{hyperref}
\usepackage{graphicx}
\usepackage{amsfonts}
\usepackage{enumitem}
\usepackage{amsmath}

\decimalpoint
\hypersetup{colorlinks=true, linkcolor=black, urlcolor=blue}
\setlength{\parindent}{0em}
\setlength{\parskip}{0.5em}
\setcounter{tocdepth}{3} % profundidad de indice
\setcounter{section}{0} % nro de section
\renewcommand{\thesubsubsection}{\thesubsection.\Alph{subsubsection}}
\graphicspath{ {images/} }

% End latex config

\begin{document}

\titulo{Final 28/07/2021}
\fecha{2do cuatrimestre 2021}
\materia{Álgebra I}
\integrante{Yago Pajariño}{546/21}{ypajarino@dc.uba.ar}

%Carátula
\maketitle
\newpage

%Indice
\tableofcontents
\newpage

% Aca empieza lo propio del documento
\section{Final 28/07/2021}

\subsection{Ejercicio 1}

Demostración usando el principio de inducción

Defino $ p(n): \begin{cases}
    a_n \in \mathbb{Z} \\
    5^n | a_n \\
    5^{n+1} \not | a_n \\
\end{cases} \forall n \in \mathbb{N}_0$

Luego $p(n)$ será verdadero si las tres condiciones son verdaderas.

\textbf{Casos base n = 0; n = 1}

$ p(0): \begin{cases}
    a_0 \in \mathbb{Z} \iff 1 \in \mathbb{Z} \\
    5^0 | a_0 \iff 1 | 1 \\
    5^{0+1} \not | a_0 \iff 5 \not | 1 \\
\end{cases} $

Luego $ p(0) $ es verdadero.

$ p(1): \begin{cases}
    a_1 \in \mathbb{Z} \iff -5 \in \mathbb{Z} \\
    5^1 | a_1 \iff 5 | 5 \\
    5^{1+1} \not | a_1 \iff 25 \not | 5 \\
\end{cases} $

Luego $ p(1) $ es verdadero.

\textbf{Paso inductivo}

Dado $ h \geq 0 $ quiero probar que $ p(h) \wedge p(h+1) \implies p(h+2) $

HI: $ p(h): \begin{cases}
    a_h \in \mathbb{Z} \\
    5^h | a_h \\
    5^{h+1} \not | a_h \\
\end{cases} $
$ p(h+1): \begin{cases}
    a_{h+1} \in \mathbb{Z} \\
    5^{h+1} | a_{h+1} \\
    5^{h+2} \not | a_{h+1} \\
\end{cases} $

Luego busco probar que $ p(h+2) $ es verdadero $ \iff \begin{cases}
    a_{h+2} \in \mathbb{Z} \\
    5^{h+2} | a_{h+2} \\
    5^{h+3} \not | a_{h+2} \\
\end{cases} $

Pero,
\begin{align*}
    a_{h+2} = \frac{a_{h+1}^3}{5} + 75a_h
\end{align*}

Y ahora pruebo cada condición en particular.
\begin{align*}
    a_{h+2} \in \mathbb{Z} &\iff \frac{a_{h+1}^3}{5} + 75a_h \in \mathbb{Z} \\
    &\iff \frac{a_{h+1}^3}{5} \in \mathbb{Z} \\
    &\iff 5|a_{h+1}^3 \\
\end{align*}
Pero por HI,
\begin{align*}
    5^{h+1} | a_{h+1} &\iff a_{h+1} = 5^{h+1} \cdot k \\
    &\iff a_{h+1} = 5^h \cdot 5 \cdot k \\
    &\iff a_{h+1} = (5^h \cdot k) \cdot 5 \\
    &\iff 5|a_{h+1} \\
\end{align*}
Y por propiedades de divisibilidad, $ 5|a \implies 5|\sigma a, \forall \sigma \in \mathbb{Z} $, en particular, $ 5 | a_{h+1}^3 $ como se quería probar.

Luego $ a_{h+2} \in \mathbb{Z} $

Ahora quiero saber si $ 5^{h+2} | a_{h+2} $

Por definición $ a_{h+2} = \frac{a_{h+1}^3}{5} + 75a_h $

Luego quiero probar que $ 5^{h+2} | \frac{a_{h+1}^3}{5} + 75a_h $

Pero $ 75a_h = 25.3.a_h $ y se que $ 5^h | a_h $, luego
\begin{align*}
    a_h &= 5^h.p_{1}^h...p_{m}^{r_{m}} \\
    7a_h &= 5^2.5^h.p_1^h...p_m^{r_m} \\
    7a_h &= 5^{h+2}.p_1^h...p_m^{r_m} \\
    \implies& 5^{h+2}|75a_h \\
\end{align*}
Entonces, necesito probar que $ 5^{h+2} | \frac{a_{h+1}^3}{5} $

Pero se que
\begin{align*}
    5^{h+1} | a_{h+1} &\iff a_{h+1} = 5^{h+1} . p_1^{r_1}...p_m^{r_m} \\
    &\iff a_{h+1}^3 = 5^{3(h+1)} . p_1^{3r_1}...p_m^{3r_m} \\
    &\iff \frac{a_{h+1}^3}{5} = 5^{3(h+1) - 1} . p_1^{3r_1}...p_m^{3r_m} \\
\end{align*}
Luego $ 5^{h+2} | \frac{a_{h+1}^3}{5} \iff 3(h+1) - 1 \geq h+2 \iff 3h+3-1 \geq h+2 \iff 2h \geq 0 $

Que es verdadero, $ \forall h \geq 0 $

Luego $ 5^{h+2} | a_{h+2} $ como se quería probar.

Por ultimo quiero ver que $ 5^{h+3} \not | a_{h+2} \iff 5^{h+3} \not | \frac{a_{h+1}^3}{5} + 75a_h $

Por inciso anterior, $ 5^{h+3} | \frac{a_{h+1}^3}{5} $ pero $ 5^{h+3} \not | 5^{h+2}.3.p_1^{r_1}...p_m^{r_m} $

Y como divide a uno de los sumandos pero no al otro, $ 5^{h+2} | a_{h+2} $ como se quería probar.

Por lo tanto queda probado el paso inductivo y así $ p(n) $ es verdadero $ \forall n \in \mathbb{N}_0 $

\subsection{Ejercicio 3}

Busco el polinomio $f$ de grado mínimo mónico tal que:
\begin{itemize}
    \item (a) f tiene copmo raíz a alguna raíz sexta de la unidad
    \item (b) $ (f:f') = x^2(x^2 + 1) $
    \item (c) $ x- \sqrt[]{3} | f $ en $ \mathbb{R}[x] $
\end{itemize}

Por (a) se que f tiene raíz $ a / a^6 = 1 $, pero $ f \in \mathbb{Q}[x] \implies a \in \mathbb{Q} \iff a =\pm 1 $

Por (b) se que $ x^2 | f' \implies x^3 | f $ y $ x^2 + 1 | f' \iff (x^2 + 1)^2 | f $

Luego $ f = (x \pm 1)x^3(x^2+1)^2 $

Por (c) $ x- \sqrt[]{3} | f $ pero como $ f \in \mathbb{Q} \implies (x^2 - 3) |f $

Así, $ f = (x-1)x^3(x^2+1)^2(x^2-3) $ cumple lo pedido

\subsection{Ejercicio 4}

Por lema de Gauss, si $f$ tiene una raíz racional, $ \frac{c}{d} \implies c|p \wedge d | 1 $

Luego los posibles candidatos son $ \{ \pm 1, \pm p \} $

Evaluando obtengo que $ p = 17 $ es el único primo tal que $f$ admite una raíz racional positiva $a = 1$

Busco ahora la factorización de f.

Se que $ (x-1) | f $

Usando Ruffini, $ f = (x-1)(x^3-x^2-17x+17) $

Defino $ g = x^3-x^2-17x+17 $ y a ojo veo que $ g(1) = 0 \implies (x-1) |g $

Usando Ruffini, $ g = (x-1)(x^2-17) $

Defino $ h = x^2-17 $ y busco sus raíces usando la resolvente cuadrática.

Luego $ h = (x-\sqrt[]{17})(x+\sqrt[]{17}) $

Usando todo lo hallado armo las factorizaciones.

\begin{itemize}
    \item $ f = (x-1)(x-1)(x-\sqrt[]{17})(x+\sqrt[]{17}) $ es la factorización en $ \mathbb{R}[x]; \mathbb{C}[x] $
    \item $ f = (x-1)(x-1)(x^2-17) $ es la factorización en $ \mathbb{Q}[x] $
\end{itemize}

\end{document}

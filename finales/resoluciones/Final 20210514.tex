\documentclass{article}
\usepackage{ifthen}
\usepackage{amssymb}
\usepackage{multicol}
\usepackage{graphicx}
\usepackage[absolute]{textpos}
\usepackage{amsmath, amscd, amssymb, amsthm, latexsym}
\usepackage{xspace,rotating,dsfont,ifthen}
\usepackage[spanish,activeacute]{babel}
\usepackage[utf8]{inputenc}
\usepackage{pgfpages}
\usepackage{pgf,pgfarrows,pgfnodes,pgfautomata,pgfheaps,xspace,dsfont}
\usepackage{listings}
\usepackage{multicol}
\usepackage{todonotes}
\usepackage{url}
\usepackage{float}
\usepackage{framed,mdframed}
\usepackage{cancel}

\usepackage[strict]{changepage}


\makeatletter


\newcommand\hfrac[2]{\genfrac{}{}{0pt}{}{#1}{#2}} %\hfrac{}{} es un \frac sin la linea del medio

\newcommand\Wider[2][3em]{% \Wider[3em]{} reduce los m\'argenes
\makebox[\linewidth][c]{%
  \begin{minipage}{\dimexpr\textwidth+#1\relax}
  \raggedright#2
  \end{minipage}%
  }%
}


\@ifclassloaded{beamer}{%
  \newcommand{\tocarEspacios}{%
    \addtolength{\leftskip}{4em}%
    \addtolength{\parindent}{-3em}%
  }%
}
{%
  \usepackage[top=1cm,bottom=2cm,left=1cm,right=1cm]{geometry}%
  \usepackage{color}%
  \newcommand{\tocarEspacios}{%
    \addtolength{\leftskip}{3em}%
    \setlength{\parindent}{0em}%
  }%
}

\usepackage{caratula}
\usepackage{enumerate}
\usepackage{hyperref}
\usepackage{graphicx}
\usepackage{amsfonts}
\usepackage{enumitem}
\usepackage{amsmath}

\decimalpoint
\hypersetup{colorlinks=true, linkcolor=black, urlcolor=blue}
\setlength{\parindent}{0em}
\setlength{\parskip}{0.5em}
\setcounter{tocdepth}{3} % profundidad de indice
\setcounter{section}{0} % nro de section
\renewcommand{\thesubsubsection}{\thesubsection.\Alph{subsubsection}}
\graphicspath{ {images/} }

% End latex config

\begin{document}

\titulo{Final 14/05/2021}
\fecha{2do cuatrimestre 2021}
\materia{Álgebra I}
\integrante{Yago Pajariño}{546/21}{ypajarino@dc.uba.ar}

%Carátula
\maketitle
\newpage

%Indice
\tableofcontents
\newpage

% Aca empieza lo propio del documento
\section{Final 14/05/2021}

\subsection{Ejercicio 1}

Defino $ p(n): \frac{4^n}{n+1} < \binom{2n}{n}; \forall n \geq 2 $

\textbf{Caso base n = 2}

\begin{align*}
    p(2)&: \frac{4^2}{2+1} < \binom{2.2}{2} \\
    p(2)&: \frac{16}{3} < \binom{4}{2} \\
    p(2)&: \frac{16}{3} < \frac{4!}{2!2!} \\
    p(2)&: \frac{16}{3} < 6 \\
\end{align*}
$ p(2) $ es verdadero.

\textbf{Paso inductivo}

Dado $ h \geq 2 $, quiero probar que $ p(h) \implies p(h+1) $

HI: $ \frac{4^h}{h+1} < \binom{2h}{h} \implies \frac{4^h}{h+2} < \binom{2h}{h}\frac{h+1}{h+2} $

QpQ: $ \frac{4^{h+1}}{h+2} < \binom{2(h+1)}{h+1} $

Pero,
\begin{align*}
    \frac{4^{h+1}}{h+2} &= \frac{4^h}{h+2} \cdot 4 \\
    &\leq \binom{2h}{h}\frac{h+1}{h+2} \cdot 4 \\
\end{align*}

Luego alcanza probar que,
\begin{align*}
    \binom{2h}{h}\frac{h+1}{h+2} \cdot 4 &< \binom{2(h+1)}{h+1} \\
    \frac{(2h)!}{h!h!} \cdot \frac{h+1}{h+2} \cdot 4 &< \frac{(2h+2)!}{(h+1)!(h+1)!} \\
    \frac{(2h)!(h+1)4}{h!h!(h+2)} &< \frac{(2h+2)(2h+1)(2h)!}{(h+1)h!(h+1)h!} \\
    \frac{(h+1)4}{h+2} &< \frac{(2h+2)(2h+1)}{(h+1)^2} \\
    (h+1)^3 \cdot 4 &< 2(h+1)(2h+1)(h+2) \\
    (h+1)^2 \cdot \frac{4}{2} &< (2h+1)(h+2) \\
    2h^2 + 4h + 2 &< 2h^2 + 4h + h + 2 \\
    0 &< h \\
\end{align*}
Dado que $ h \geq 2 $ qued probado el paso inductivo.

Luego $ p(n) $ es verdadero, $ \forall n \in \mathbb{N}_{\geq 2}$

\subsection{Ejercicio 2}

Tengo el conjunto $ X = P(\{ 1,2,3,...,12 \}) $. Se define $R$ relación tal que $ ARB \iff \#\text{pares}(A) = \#\text{pares}(B) $

\subsubsection{Pregunta i}

Voy a probar cada propiedad de la relación de equivalencia por separado.

\textbf{Reflexividad}

Por definición de reflexividad, $ R $ es reflexiva $ \iff \forall A \in X: ARA $

Por definición de la relación, $ ARA \iff \#\text{pares}(A) = \#\text{pares}(A) $

Dado que $ A = A $, en particular tienen los mismos elementos pares, luego $ R $ es reflexiva.

\textbf{Simetría}

Por definición de simetría, $ R $ es simétrica $ \iff \forall (A,B) \in X^2: ARB \implies BRA $

Por definición de la relación, $ ARB \iff \#\text{pares}(A) = \#\text{pares}(B) $

Y quiero probar que $ BRA \iff \#\text{pares}(B) = \#\text{pares}(A) $

Pero,
\begin{align*}
    ARB &\iff \#\text{pares}(A) = \#\text{pares}(B) \\
    &\iff \#\text{pares}(B) = \#\text{pares}(A) \\
    &\iff BRA \\
\end{align*}
Luego $R$ es simétrica.

\textbf{Transitividad}

Por definición de transitividad, $ R $ es transitiva $ \iff \forall (A,B,C) \in X^3: (ARB \wedge BRC) \implies ARC $

Por definición de la relación, 
\begin{align*}
    ARB &\iff \#\text{pares}(A) = \#\text{pares}(B) \\
    BRC &\iff \#\text{pares}(B) = \#\text{pares}(C) \\
\end{align*}
Luego,
\begin{align*}
    ARB \wedge BRC &\iff \#\text{pares}(A) = \#\text{pares}(B) \wedge \#\text{pares}(B) = \#\text{pares}(C) \\
    &\implies \#\text{pares}(A) = \#\text{pares}(C) \\
    &\implies ARC
\end{align*}
Luego $R$ es transitiva.

Dado que $R$ es reflexiva, simétrica y transitiva; $R$ es una relación de equivalencia.

\subsubsection{Pregunta ii}

Veo que por definición de la relación, lo que determina que un conjunto pertenezca a una clase de equivalencia es la cantidad de pares que contenga.

Luego existen 7 clases de equivalencia: clases con elementos que contienen 0, 1, 2, 3, 4, 5, 6 pares.

Luego la clase del $ \#\{ \text{3 pares} \} = \binom{6}{3} \cdot 2^6 = 1280 $ tiene más de 1000 elementos como se quería probar.

\subsection{Ejercicio 3}

Defino $ d = (a^{60} + 6:560) $ y se que $ 560 = 7.2^4.5 $

Luego la factorización en primos de $d$ será:

$ d = 2^i \cdot 5^j \cdot 7^k $ con $ \begin{cases}
    0\leq i \leq 4 \\
    0\leq j \leq 1 \\
    0\leq k \leq 1 \\
\end{cases} $

Estudio cada primo en particular.

\textbf{Caso p = 5}
\begin{align*}
    5 | a^{60} + 6 &\iff a^{60} + 6 \equiv 0 (5) \\
    &\iff a^{60} \equiv 4 (5) \\
    &\iff \begin{cases}
        0 \equiv 4(5) & 5|a \\
        1 \equiv 4(5) & 5 \not | a \text{ por PTF} \\
    \end{cases}
\end{align*}
En ambos casos se llega a un absurdo, luego $ j = 0 $

\textbf{Caso p = 7}
\begin{align*}
    7 | a^{60} + 6 &\iff a^{60} + 6 \equiv 0 (7) \\
    &\iff a^{60} \equiv 1 (7) \\
    &\iff \begin{cases}
        0 \equiv 1(7) & 7|a \\
        1 \equiv 1(7) & 7\not |a \text{ por PTF} \\
    \end{cases} \\
\end{align*}
Luego $ k = 0 \vee k = 1 $

\textbf{Caso p = 2}

Si $ a \equiv 0(2) \iff a = 2k $,
\begin{align*}
    a^{60} + 6 &\equiv 0^{60} + 6 \equiv 0 (2) \\
    a^{60} + 6 &\equiv (2k)^{2^{30}} + 6 \equiv 2 (4) \\
\end{align*}
Y si no es divisible por 4 tampoco lo es por 8 y por 16

Si $ a \equiv 1(2) $,
\begin{align*}
    a^{60} + 6 &\equiv 1^{60} + 6 \equiv 1 (2) \\
\end{align*}
Luego si no es divisible por 2, tampoco lo será por 5, 8, 16

Luego $ i = 0 \vee i = 1 $

Por lo tanto, los posibles MCD son
\begin{itemize}
    \item $ k = 0 \wedge i = 0 \implies d = 1 $
    \item $ k = 0 \wedge i = 1 \implies d = 2 $
    \item $ k = 1 \wedge i = 0 \implies d = 7 $
    \item $ k = 1 \wedge i = 1 \implies d = 14 $
\end{itemize}

\subsection{Ejercicio 5}

P tiene una raíz imaginaria pura $ \iff \exists a \in \mathbb{R}: P(ai) = 0 $

\begin{align*}
    P(ai) = 0 &\iff (ai)^6 + (ai)^5 + 5(ai)^4 + 4(ai)^3 + 8(ai)^2 + 4(ai) + 4 = 0 \\
    &\iff -a^6 + a^5i + 5a^4 - 4a^3i - 8a^2 + 4ai + 4 = 0 \\
    &\iff (-a^6 + 5a^4 - 8a^2 + 4) + (a^5 - 4a^3 + 4a)i = 0 \\
    &\iff \begin{cases}
        -a^6 + 5a^4 - 8a^2 + 4 = 0 \\
        a^5 - 4a^3 + 4a = 0
    \end{cases} \\
\end{align*}
De donde se obtiene que $ a = \sqrt[]{2} $

Luego $ \sqrt[]{2}i $ es raíz de P $ \iff -\sqrt[]{2}i $ es raíz de P.

Por lo tanto $ (x-\sqrt[]{2}i)(x+\sqrt[]{2}i)|P \iff (x^2 + 2)|P $

Usando el algoritmo de división de polinomios, $ P = (x^2 + 2)(x^4 + x^3 + 3x^2 + 2x + 2) $

Y por enunciado se que son raíces múltiples, luego se que $ (x^2 + 2) | (x^4 + x^3 + 3x^2 + 2x + 2) $

Usando el algoritmo de división, $ P = (x^2 + 2)^2(x^2 + x + 1) $

Defino $ g = x^2 + x + 1 $ y busco sus raíces utilizando la resolvente cuadrática.

Obtengo que $ g = (x-(-\frac{1}{2} + \frac{\sqrt[]{3}}{2}i))(x-(-\frac{1}{2} - \frac{\sqrt[]{3}}{2}i)) $

Por lo tanto,
\begin{itemize}
    \item $ P = (x-\sqrt[]{2}i)^2(x+\sqrt[]{2}i)^2(x-(-\frac{1}{2} + \frac{\sqrt[]{3}}{2}i))(x-(-\frac{1}{2} - \frac{\sqrt[]{3}}{2}i)) $ es la factorización en $ \mathbb{C}[x] $
    \item $ P = (x^2 + 2)^2(x^2 + x + 1) $ es la factorización en $ \mathbb{Q}[x]; \mathbb{R}[x] $
\end{itemize}

Las raíces de P son sus multiplicidades son:
\begin{itemize}
    \item $ mult(\sqrt[]{2}i, f) = 2 $
    \item $ mult(-\sqrt[]{2}i, f) = 2 $
    \item $ mult(-\frac{1}{2} + \frac{\sqrt[]{3}}{2}i, f) = 1 $
    \item $ mult(-\frac{1}{2} - \frac{\sqrt[]{3}}{2}i, f) = 1 $
\end{itemize}

\end{document}

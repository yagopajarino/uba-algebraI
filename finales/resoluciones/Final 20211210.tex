\documentclass{article}
\usepackage{ifthen}
\usepackage{amssymb}
\usepackage{multicol}
\usepackage{graphicx}
\usepackage[absolute]{textpos}
\usepackage{amsmath, amscd, amssymb, amsthm, latexsym}
\usepackage{xspace,rotating,dsfont,ifthen}
\usepackage[spanish,activeacute]{babel}
\usepackage[utf8]{inputenc}
\usepackage{pgfpages}
\usepackage{pgf,pgfarrows,pgfnodes,pgfautomata,pgfheaps,xspace,dsfont}
\usepackage{listings}
\usepackage{multicol}
\usepackage{todonotes}
\usepackage{url}
\usepackage{float}
\usepackage{framed,mdframed}
\usepackage{cancel}

\usepackage[strict]{changepage}


\makeatletter


\newcommand\hfrac[2]{\genfrac{}{}{0pt}{}{#1}{#2}} %\hfrac{}{} es un \frac sin la linea del medio

\newcommand\Wider[2][3em]{% \Wider[3em]{} reduce los m\'argenes
\makebox[\linewidth][c]{%
  \begin{minipage}{\dimexpr\textwidth+#1\relax}
  \raggedright#2
  \end{minipage}%
  }%
}


\@ifclassloaded{beamer}{%
  \newcommand{\tocarEspacios}{%
    \addtolength{\leftskip}{4em}%
    \addtolength{\parindent}{-3em}%
  }%
}
{%
  \usepackage[top=1cm,bottom=2cm,left=1cm,right=1cm]{geometry}%
  \usepackage{color}%
  \newcommand{\tocarEspacios}{%
    \addtolength{\leftskip}{3em}%
    \setlength{\parindent}{0em}%
  }%
}

\usepackage{caratula}
\usepackage{enumerate}
\usepackage{hyperref}
\usepackage{graphicx}
\usepackage{amsfonts}
\usepackage{enumitem}
\usepackage{amsmath}

\decimalpoint
\hypersetup{colorlinks=true, linkcolor=black, urlcolor=blue}
\setlength{\parindent}{0em}
\setlength{\parskip}{0.5em}
\setcounter{tocdepth}{3} % profundidad de indice
\setcounter{section}{0} % nro de section
\renewcommand{\thesubsubsection}{\thesubsection.\Alph{subsubsection}}
\graphicspath{ {images/} }

% End latex config

\begin{document}

\titulo{Final 10/12/2021}
\fecha{2do cuatrimestre 2021}
\materia{Álgebra I}
\integrante{Yago Pajariño}{546/21}{ypajarino@dc.uba.ar}

%Carátula
\maketitle
\newpage

%Indice
\tableofcontents
\newpage

% Aca empieza lo propio del documento
\section{Final 10/12/2021}

\subsection{Ejercicio 1}

Sea $ V = \{ 1,2,...,499,500 \} $. Se define la relación $R$ en $ P = P(V) \backslash \emptyset $ como:
\begin{align*}
    ARB \iff min(A) = min(B) \wedge max(A) = max(B)
\end{align*}

\subsubsection{Demostración clase de equivalencia}

Me piden probar que $R$ es una relación de equivalencia. Por definición, una relación es de equivalencia si es reflexiva, simétrica y transitiva. Pruebo cada propuedad por separado.

\textbf{Reflexividad}

Por definición, $R$ es reflexiva $ \iff \forall A \in P: ARA $

Por definición de la relación,
\begin{align*}
    ARA \iff min(A) = min(A) \wedge max(A) = max(A)
\end{align*}
Dado que $ A = A $ en particular, tienen el mismo mínimo y el mismo máximo. Luego $R$ es reflexiva.

\textbf{Simetría}

Por definición, $R$ es simétrica $ \iff \forall (A,B) \in P^2: ARB \implies BRA $

Por definición de la relación,
\begin{align*}
    ARB &\iff min(A) = min(B) \wedge max(A) = max(B) \\
    &\iff min(B) = min(A) \wedge max(B) = max(A) \\
    &\iff BRA \\
\end{align*}
Por lo tanto $ ARB \implies BRA $ como se quería probar, luego $R$ es simétrica.

\textbf{Transitividad}

Por definición, $R$ es transitiva $ \iff \forall (A,B,C) \in C^3: (ARB \wedge BRC) \implies ARC $

Por definición de la relación,
\begin{align*}
    ARB &\iff min(A) = min(B) \wedge max(A) = max(B) \\
    BRC &\iff min(B) = min(C) \wedge max(B) = max(C) \\
\end{align*}
Pero,
\begin{align*}
    min(A) = min(B) \wedge min(B) = min(C) &\implies min(A) = min(B) = min(C) \\
    &\implies min(A) = min(C) \\
\end{align*}
Y,
\begin{align*}
    max(A) = max(B) \wedge max(B) = max(C) &\implies max(A) = max(B) = max(C) \\
    &\implies max(A) = max(C) \\
\end{align*}
Por lo tanto,
\begin{align*}
    min(A) = min(C) \wedge max(A) = max(C) \iff ARC
\end{align*}
Luego $R$ es una relación transitiva.

Por lo tanto, $R$ es una relación de equivalencia, dado que es una relación relfexiva, simétrica y transitiva.

\subsubsection{Cardinal de la clase $ x =\{ 1,100 \} $}

Busco todos los $ B \in P: XRB $

Por definición, 
\begin{align*}
    XRB &\iff min(X) = min(B) \wedge max(X) = max(B) \\
    &\iff 1 = min(B) \wedge 100 = max(B) \\
\end{align*}
Por lo tanto, busco todos los $B \in P(V) \backslash \emptyset: min(B) = 1 \wedge max(B) = 100 $ 

Sabiendo que $ V = \{ 1,2,...,499,500 \} $ tengo que contar todos los subconjuntos de $V$ que poseen al $1$, poseen al $100$ y no poseen ningún $ a \in V: a > 100 $

Por lo tanto,
\begin{itemize}
    \item $ 1 $ tiene $ 1 $ posibilidad $ \implies 1 $
    \item $ 2-99 $ tienen $ 2 $ posibilidades $ \implies 2^{98} $
    \item $ 100 $ tiene $ 1 $ posibilidad $ \implies 1 $
    \item $ 101-500 $ tienen $ 1 $ posibilida $ \implies 1 $
\end{itemize}
Por lo tanto, habrá $ 1.2^{98}.1.1 = 2^{98} $ conjuntos.

Rta.: $ \# \overline{\{ 1,100 \}} = 2^{98} $

\subsubsection{Cadinal de la clase $ Y = \{ 50 \} $}

Busco todos los $ C \in P: YRC $

Por definición,
\begin{align*}
    YRC &\iff min(Y) = min(C) \wedge max(Y) = max(C) \\
    &\iff 50 = min(C) \wedge 50 = max(C)
\end{align*}
Por lo tanto, busco todos los $C \in P(V) \backslash \emptyset: min(C) = 50 \wedge max(C) = 50 $ 

Pero el único conjunto que cumple ambas en simultáneo es $ C = \{ 50 \} $ y por lo tanto,

Rta.: $ \# \overline{\{ 50 \}} = 1 $

\subsubsection{Cantidad de clases de equivalencia de $R$}

Para saber si un subconjunti de $V$ pertenece a una clase de equivalencia, alcanza con observar el mínimo y el máximo del subconjunto.

Separo en dos casos, 

(1) Clases del tipo $ \{ n \} $ con $ n \in V $ forman $ 500 $ clases distintas y no se relacionan con subconjuntos de dos o más elementos.

(2) Clases del tipo $ \{ a_1, ..., a_r \} \wedge 2 \leq r \leq 500 \wedge a_i \in V $ 

En este caso voy a tener $ 500 - a_1 $ clases. Por ejemplo
\begin{itemize}
    \item $ a_1 = 1 \implies \overline{\{ 1,500 \}}, \overline{\{ 1,499 \}}, \overline{\{ 1,498 \}}... \implies 499 $ clases.
    \item $ a_1 = 2 \implies \overline{\{ 2,500 \}}, \overline{\{ 2,499 \}}, \overline{\{ 3,498 \}}... \implies 498 $ clases.
    \item $ a_1 = 499 \implies \overline{\{ 499,500 \}} \implies 1 $ clase.
\end{itemize}
Luego habrá $ 500 + \sum_{i = 1}^{499}i = 500 + \frac{499.500}{2} = 125250 $ clases de equivalencia en la relación $R$.

\subsection{Ejercicio 2}

Se que $ 252 = 2^2. 3^2 .7 $ y que $ 14 = 2.7 $

Luego $ (a^{255} + 10a + 1:252) = 14 \implies \begin{cases}
    2 | a^{255} + 10a + 1 \\
    4 \not | a^{255} + 10a + 1 \\
    7 | a^{255} + 10a + 1 \\
    3 \not | a^{255} + 10a + 1 \\
\end{cases} $ 

Ahora busco los $a$ que cumplen cada una de estas restricciones.

\textbf{Caso 2}
\begin{align*}
    2 | a^{255} + 10a + 1 &\iff a^{255} + 10a + 1 \equiv 0 (2) \\
    &\iff \begin{cases}
        a \equiv 0(2) \implies a^{255} + 10a + 1 \equiv 0 + 0 + 1 \not \equiv 0(2) \\
        a \equiv 1(2) \implies a^{255} + 10a + 1 \equiv 1 + 0 + 1 \equiv 0(2) \\
    \end{cases}
\end{align*}
Luego $ a \equiv 1(2) $

\textbf{Caso 4}

$ a \equiv 1(2) \implies a \equiv 1(4) \vee a \equiv 3(4) $

Luego,
\begin{itemize}
    \item $ a \equiv 1(4) \implies a^{255} + 10a + 1 \equiv 1 + 2 + 1 \equiv 0 (4) $
    \item $ a \equiv 3(4) \implies a^{255} + 10a + 1 \equiv 3^{255} + 2 + 1 \equiv 9^{112}.3 + 3 \equiv 2 (4) $
\end{itemize}
Luego $ a \equiv 3(4) $

\textbf{Caso 7}

$ 7 | a^{255} + 10a + 1 \iff a^{255} + 10a + 1 \equiv 0 (7) $

Separo en dos casos: $ 7|a $ y $ 7 \not | a $ para poder usar el PTF

\begin{itemize}
    \item $ 7 | a \implies a^{255} + 10a + 1 \equiv 0 + 0 + 1 \not \equiv 0 (7) $
    \item $ 7 \not | a \implies a^{255} + 10a + 1 \equiv (a^6)^{37}. a^3 + 10a + 1 \equiv a^3 + 3a + 1 (7) $
\end{itemize}

Por lo tanto busco los $a: a^3 + 3a + 1 \equiv 0(7)$

\begin{itemize}
    \item $ a \equiv 0(7) \implies a^3 + 3a + 1 \equiv 1(7) $
    \item $ a \equiv 1(7) \implies a^3 + 3a + 1 \equiv 5(7) $
    \item $ a \equiv 2(7) \implies a^3 + 3a + 1 \equiv 1(7) $
    \item $ a \equiv 3(7) \implies a^3 + 3a + 1 \equiv 2(7) $
    \item $ a \equiv 4(7) \implies a^3 + 3a + 1 \equiv 0(7) $
    \item $ a \equiv 5(7) \implies a^3 + 3a + 1 \equiv 1(7) $
    \item $ a \equiv 6(7) \implies a^3 + 3a + 1 \equiv 4(7) $
\end{itemize}
Por lo tanto, $ a^3 + 3a + 1 \equiv 0(7) \iff a \equiv 4(7) $

Luego $ a \equiv 4(7) $

\textbf{Caso 3}
\begin{align*}
    3 \not | a^{255} + 10a + 1 &\iff a^{255} + 10a + 1 \not \equiv 0 (3) \\
\end{align*}

\begin{itemize}
    \item $ a\equiv 0(3) \implies a^{255} + 10a + 1 \equiv 0 + 0 + 1 \not \equiv 0 (3) $
    \item $ a\equiv 1(3) \implies a^{255} + 10a + 1 \equiv 1 + 1 + 1 \equiv 0 (3) $
    \item $ a\equiv 2(3) \implies a^{255} + 10a + 1 \equiv (a^3)^{85} + 10.a + 1 \equiv (-1)^{85} + 2 + 1 \equiv -1+2+1 \not \equiv 0 (3) $
\end{itemize}
Luego $ a \equiv 0 (3) \vee a \equiv 2 (3) $

Pero estoy buscando el resto mod 252, por lo que necesito saber $ a \equiv n (9) $

Sabiendo las equivalencias mod 3, busco mod 9:
\begin{align*}
    a \equiv 0 (3) \implies a \equiv 0 (9) \vee a \equiv 3 (9) \vee a \equiv 6 (9) \\
    a \equiv 2 (3) \implies a \equiv 2 (9) \vee a \equiv 5 (9) \vee a \equiv 8 (9) \\
\end{align*}

Por lo tanto, juntando todo lo hallado,
\begin{align*}
    (a^{255} + 10a + 1:252) = 14 \iff \begin{cases}
        a \equiv 3(4) \\
        a \equiv 4(7) \\
        a \equiv 0 (9) \vee a \equiv 3 (9) \vee a \equiv 6 (9) \vee a \equiv 2 (9) \vee a \equiv 5 (9) \vee a \equiv 8 (9) \\
    \end{cases}
\end{align*}

Ahora uso el Teorema Chino del Resto para hallar la equivalencia mod 252.

Busco soluciones a los sistemas
$ S_0 = \begin{cases}
    a \equiv 3(4) \\
    a \equiv 4(7) \\
    a \equiv 0(9)
\end{cases} $
$ S_1 = \begin{cases}
    a \equiv 3(4) \\
    a \equiv 4(7) \\
    a \equiv 3(9)
\end{cases} $
$ S_2 = \begin{cases}
    a \equiv 3(4) \\
    a \equiv 4(7) \\
    a \equiv 6(9)
\end{cases} $
$ S_3 = \begin{cases}
    a \equiv 3(4) \\
    a \equiv 4(7) \\
    a \equiv 2(9)
\end{cases} $
$ S_4 = \begin{cases}
    a \equiv 3(4) \\
    a \equiv 4(7) \\
    a \equiv 5(9)
\end{cases} $
$ S_5 = \begin{cases}
    a \equiv 3(4) \\
    a \equiv 4(7) \\
    a \equiv 8(9)
\end{cases} $

Por TCR se que existe una única solución a cada sistema $ X = x_1 + x_2 + x_3 $ mod $255$

Donde,

$ x_1 $ es solución del sistema $ \begin{cases}
    a \equiv 3(4) \\
    a \equiv 0(63) \implies a = 63k \implies 63k \equiv 3(4) \iff k \equiv 1(4)
\end{cases} $

Luego $x_1 = 63$

$ x_2 $ es solución del sistema $ \begin{cases}
    a \equiv 4(7) \\
    a \equiv 0(36) \implies a = 36k \implies 36k \equiv 4(7) \implies k \equiv 4(7)
\end{cases} $

Luego $x_2 = 36.4 = 144 $

$ x_3 $ es solución del sistema $ \begin{cases}
    a \equiv n(9) \\
    a \equiv 0(28)
\end{cases} $ con $n$ el valor de mod 9 de cada $S_i$

\begin{itemize}
    \item $ a \equiv 0(9) \implies x_3 = 0 $
    \item $ a \equiv 2(9) \implies a = 28k \implies 28k \equiv 2(9) \implies k \equiv 2(9) \implies x_2 = 28.2 = 56 $
    \item $ a \equiv 3(9) \implies a = 28k \implies 28k \equiv 3(9) \implies k \equiv 3(9) \implies x_2 = 28.3 = 84 $
    \item $ a \equiv 5(9) \implies a = 28k \implies 28k \equiv 5(9) \implies k \equiv 5(9) \implies x_2 = 28.5 = 140 $
    \item $ a \equiv 6(9) \implies a = 28k \implies 28k \equiv 6(9) \implies k \equiv 6(9) \implies x_2 = 28.6 = 168 $
    \item $ a \equiv 8(9) \implies a = 28k \implies 28k \equiv 8(9) \implies k \equiv 8(9) \implies x_2 = 28.8 = 224 $
\end{itemize}

Por lo tanto $ r_{252}(a) $ serán:
\begin{itemize}
    \item $ 63 + 144 + 0 = 207 $
    \item $ 63 + 144 + 56 = 11 $
    \item $ 63 + 144 + 84 = 39 $
    \item $ 63 + 144 + 140 = 95 $
    \item $ 63 + 144 + 168 = 123 $
    \item $ 63 + 144 + 224 = 180 $
\end{itemize}

\subsection{Ejercicio 3}

\subsubsection{Pregunta i}

Defino $ f = x^2 + x + 1 $ y $ g = x^{2n} + x^n + 1 $

Se que
\begin{align*}
    f = \left( x - \left( -\frac{1}{2} + \frac{\sqrt[]{3}}{2}i \right) \right)\left( x - \left( -\frac{1}{2} - \frac{\sqrt[]{3}}{2}i \right) \right)
\end{align*}

Y se que $ w_1 = \left( -\frac{1}{2} + \frac{\sqrt[]{3}}{2}i \right) $; $ w_2 = \left( -\frac{1}{2} - \frac{\sqrt[]{3}}{2}i \right) $ son raíces cúbicas de la unidad.

Por propiedades de las raíces de la unidad, se que $ w \in G_n \implies w^k = w^{r_n(k)} $

Y también vale que $ w \in g_3 \wedge w \neq 1 \implies 1 + w + w^2 = 0 $

Por lo tanto, $ f|g \iff g(w1) = 0 \wedge g(w_2) = 0 $

\begin{itemize}
    \item $ n \equiv 0(3) \implies g(w_1) = w_1^0 + w_1^0 + 1 \neq 0 $
    \item $ n \equiv 1(3) \implies g(w_1) = w_1^2 + w + 1 = 0 $
    \item $ n \equiv 2(3) \implies g(w_1) = w_1^1 + w^2 + 1 = 0 $
\end{itemize}

Luego $ f|g \iff n \equiv 1(3) \wedge n \equiv 2(3) $

\subsubsection{Pregunta ii}
TODO

\subsection{Ejercicio 4}

$ a \in \mathbb{C} $ es una raíz con $ mult(a, f) = 5 \iff f = (x-a)^5 . q \wedge q(a) \neq 0 $

Busco el termino general de la multiplicidad de $a$ como raíz de los polinomios de la sucesión,
\begin{itemize}
    \item $ n = 1 \implies mult(a, f_1) = 5 $
    \item $ n = 2 \implies mult(a, f_2) = 7 $
    \item $ n = 3 \implies mult(a, f_3) = 9 $
    \item $ n = 4 \implies mult(a, f_4) = 11 $
    \item $ n = 5 \implies mult(a, f_5) = 13 $
\end{itemize}

Parece que la multiplicidad de $a$ como raíz de $f_n$ es de la forma $ (n+1)2 + 1 = 2n+3 $

Sin embargo, para poder asumir que estas multiplicidades son correctas, hay que verificar que,
\begin{align*}
    2n+3 &\leq 5n \\
    3 &\leq 5n-2n \\
    3 &\leq 3n \\
    1 &\leq n \iff n \in \mathbb{N} \\
\end{align*}

Dados $h, g$ polinomios con $ mult(\alpha, h) = 3 \wedge mult(\alpha, g) = 5 \implies \begin{cases}
    mult(\alpha, fg) = 8 \\
    mult(\alpha, f + g) = 3 \\
\end{cases} $

Y en el caso general, la multiplicidad de una raíz en una suma de polinomios es la de menor grado, pues
$ h = (x-\alpha)^3 \cdot p_1 \wedge p_1(\alpha) \neq 0 $ y $ g = (x-\alpha)^5 \cdot p_2 \wedge p_2(\alpha) \neq 0  $
y por lo tanto, $ h+g = (x-\alpha)^3 \cdot p_1 + (x-\alpha)^5 \cdot p_2 = (x-\alpha)^3(p_1 + (x-\alpha)^2p_2) $
donde $ (x-a) \not | (p_1 + (x-\alpha)^2p_2) $

Usando esta propiedad, voy a probar por inducción que $ mult(a, f_n) = 2n+3 $

Defino $ p(n): mult(a, f_n) = 2n+3; \forall n \in \mathbb{N} $

\textbf{Caso base n = 1}

\begin{align*}
    p(1): &mult(a, f_1) = 2.1+3 \\
    &mult(a, f_1) = 5 \\
\end{align*}
Luego $ p(1) $ es verdadero.

\textbf{Paso inductivo}

Quiero probar que dado $ h \geq 1: p(h) \implies p(h+1) $

HI: $ mult(a, f_h) = 2h+3 \iff f_h = (x-a)^{2h+3} \cdot q_2 \wedge q_2(a) \neq 0 $

Qpq: $ mult(a, f_{h+1}) = 2(h+1)+3 $

Pero,
\begin{align*}
    f_{h+1} &= (x-a)^2f_h + f^{h+1} \\
    &= (x-a)^2f_h + ((x-a)^5q)^{h+1} \\
    &= (x-a)^2f_h + (x-a)^{5(h+1)} \cdot q^{h+1} \\
    &= (x-a)^2 \cdot (x-a)^{2h+3} \cdot q_2 + (x-a)^{5(h+1)} \cdot q^{h+1} \\
    &= (x-a)^{2h+5} (q_2 + (x-a)^{5h+5-2h-5} \cdot q^{h+1}) \\
    &= (x-a)^{2h+5} (q_2 + (x-a)^{3h} \cdot q^{h+1}) \\
\end{align*}
Luego $ mult(a, f_{h+1}) = 2h+5 $ pues $ q_2(a) \neq 0 $ como se quería probar.

Por lo tanto $ p(h) \implies p(h+1); \forall h \geq 1 $

Luego $ p(n) $ es verdadero, $ \forall n \in \mathbb{N} $

Y así $ mult(a, f_n) = 2n+3; \forall n \in \mathbb{N} $

\end{document}